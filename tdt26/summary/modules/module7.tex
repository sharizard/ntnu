\section{Module 7}
Krav er ofte forstått som hva et system skal gjøre, enn hvordan. Det trengs modelleringsteknikker for krav for å håndtere kunnskap og begrunning i kravfasen. De fleste modeller som eksisterer idag brukes sent i kravfasen, som fokuserer på komplett, konsistent og automatisk verifisering av krav. Tidlig fasen fokuserer på modellere og analysere aktørers interesser.

Krav er viktig for suksess til et system. Må fokusere på tidligere steg i utviklingsprosessen siden datasystemer spiller en viktig rolle i forretningen.

Førstegangs krav fra kunden handler ofte om hva systemet skal gjøre, men er ambigiøst, ukomplett, ukonsistent, ikke-formelt. Derfor har mange rammeverker og krav-språk blitt definert for å håndtere dette. Målet er å opprette et krav-dokument som utviklere skal følge, slik at systemet er spesifiksert. 

Tidlig fase i kravdefinering inkluderer hvordan systemet skal oppnå forrretningsmål, hvorfor systemet trengs, hva slags alternativer finnes, hva er implikasjoner for ulike aktører, hvordan aktørers interesser og bekymringer kan tas opp. Må forstå "hvorfor" som ligger i systemets krav, enn "hva" systemet skal gjøre.

Viktig fordi:
- Systemutvikling inneholder mange antagelser. Det sies at de fleste prosjekter failer fordi man ikke forstår domenet godt nok. Man er nødt til å skjønne interesser, prioriteringer og ferdigheter fra aktører osv.
- Brukere trenger hjelp med første krav. Et rammeverk trengs for å hjelpe utviklere i å skjønne hva brukere vil ha og hjelpe dem med å forstå hva systemet kan tilby. Mange systemet har ikke klart det.
- 
