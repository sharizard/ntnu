\part{Teoretical background}
\chapter{Technical Debt}


\section{Introduction}
Techincal debt is all about unfinished, unecessary complex, and outdated solutions which hinders us when it comes to maintencance of the solution in an effective way. 

Skjer når snarveies tar for å løse et IT-problem i stor grad fra kortsiktige hensyn til kostnad og tidsbruk. Medfører problemer ifm. Gjennomføring av endringer og nyutvikling hvis markedskravene endrer seg. Et økende globalt problem. 







------------------------------

You're currently working on a project. It's short time for delivery, and yet many functionality which needs to be implemented. You have two choices when implementing a functionality. You can deliver a less good solution (quick and dirty) where you'll deliver the functionality in time, but we trade off design and quality, which makes operation, management and maintenance hard to do. The other way is a much cleaner result, but in exhange of more time to put the functionality in place. This is what technical debt is, when shortcuts are taken to solve problems trading off quality and design. Technical debt is a big problem today. In 2010, the total debt was around 500 millions dollar, while it's estimated that in 2015 the technical debt will be around 1 billion dollar.

In order to understand and investigate the consequences of technical debt, one needs to know what technical debt is, the reasons for it, and how it can be eliminated.

\section{Types of technical debt}
The concept of technical debt was first introduced by Ward Cunningham\cite{cunningham}. Like financial debt, the technical debt incurs interest payments, which comes in the form of extra time and money, whenever the software needs to be refactored or redesigned\cite{fowler}. This connects the term technical debt to the problem of "deciding when and how to refactor a system to improve its structure as a basis for future evolution".

McConnels defines two categories based on how they are incurred, intentionally or unintentionally. The unintentional category includes debt that comes from doing a poor job. For instance, uninntentional debt might be when a junior software developer writes bad code due to lack of knowledge and experience. Intentional debt occurs when an organization makes a decision to optimize for the present rather than the future. An example is when the project release must be done on time, or else there wont be a next release. This leads to bad decisions, like taking a shortcut to solve a problem, and reconcile the problem after shipment

Fowlers presents a formal explanation of how techincal debt can occur. He categories technical debt into different debt types, in which he calls "Technical Debt Quadrant". As seen in the figure, the debt is grouped into four categories: reckless deliberate/inadvertent and prudent deliberate/inadvertent. 

Krutchen divides technical debt into two categories. Visible, debt that is visible for everyone. It containts elements such as new functionality to add and defects to fix. Invisible is the other category, debt that is only visible to software developers.

\subsection{Organizational debt}
Organazional debt

\section{Why does it occur}
When you Developers often wants to deliver something in time, and shortcuts is often used. However, technical debt might occur already in the requirement specification phase in software development project. 

People often sees technical debt as bad code design, but in practice, technical debt is much more than that. Like a financial loan, technical debt incurs interests as well, which can be quite dangerous. Example on interests might be lowr pace of development, low competitiveness, security flaws on the system, loss of developers and their expertise, poor internal collaboration environment, dissatisfied customers and loss of market share. 

Technical debt may be caused by several factors:
- Work processes: Software development methology, can some tasks be automized (with a deploy script), is tests written after bug fixing, do you map and document shortcuts you take, is there any plans for techincal debt management later, is it important to implement new functionality or to make sure that the existing ones work propertly?
- People (knowledge and capacity): Do you need some individuals to finish a task, Do you have the right people for the job, is enough training given to new people, what happens if you need someone who's on vacation/change project/is sick or something. You should keep this in mind and make a plan on how knowledge is transferred. Techcnical debt can be the reason for poor motivation and productivity which causes you to work poorly. 
- Technology: Is solutions hard to integrate with other solutions, is all the systems out there compatible with newer technology, is there any outdated or duplicated code in the system, is all the systems secure, is the solutions old, or user friendly, is there some code which is hard to maintencance. 
- Collaboration in the organization: Commuinication between developers and requirement people. You often get a list with requirements, is the list understandable? Do we work with a backlog with tasks that should have been solved long time ago, but not which is not "actual" now?

Developers might not care about the product because they don't feel that they "own" what's being made. They get told what do to, but not more than that. Can't make requirements.


Operating technical debt might be to maintenance and manage existing code rather than implementing new functionality. It is important to keep track of the technical debt, and incur interest payments, before it makes troubles for you. Do do that, you could for example set up a plan for repayment which tells you something about how the debt shall be repayed. Scrum can be used to do this for example, where you split the repayment plan to smaller parts where you estimate and prioritize tasks. It is important to remember that taking too much loan might cause problems. As mentioned ealier, technical debt can be seen as taking a financial loan according to Cunningham. The loan has to be repaid with interests. Technical debt uses time and effort as repayment. It is acceptable to take a loan, but it should be controlled. Do not take loan than what you are able to handle. Think with your head.

"Kort tid til leveranse, alt for mye å gjøre. Levere en mindre god løsning i tide (quick and dirty), eller gjennomarbeidet løsning hvor vi går over tidsfristen. Quick and dirty gjør at vi leverer i tide, men på bekosnitng av design og kvalitet, som fører til at drift, forvaltning og vedlikeholdt blir komplekst. ""



\section{Technical debt in Industry}
Technical debt today is connected with many differnet aspects in the software development process, like documentation debt, requirements debt, architecture debt etc. 

Using sequential design processes in software development processes to develop big software is often a failure. Requirements are specified in the beginning, and rest of the steps in this model has to follow these requirements which can't be changed. There's no benefits using this model for big softwares as technology and business requirements always change. 

This is why agile methods was made, where change and feedback is important. However, one big problem with agile methods is that developets often wants to focus on implementing new fucntionality, and have very poor focus on design, code quality, testing, which again leads to technical debt. 

There are multiple classification on techincal debt, which can occur everywhere in the software development life cycle. Like documentation debt, design debt, code, testing etc. 

Architecture violations might hiundter future development as it might be hard to extend.



Klinger

Codabux




























\chapter{Embedded Systems}
\section{Introduction}
Embedded systems is a combination of hardware and software to perform a specific task. Embedded systems has a big role today as  


\section{Embedded system software111}


Embedded software er en slags programvare for innebygde systemer. Disse programvarene er spesialisert for en type hardware (som den ligger og kjører på). Disse programvarene har derfor spesifikke begrensninger når det kommer til run-time, som minnebruk, prosesseringskraft osv.

ES har en store rolle idag siden embedded systems utvikles stadig, spesielt med IoT som en trend nå.

Problem: Har en lang livsstid som gjør det utfordrenede å vedlikeholde gamle systemer kontra utviling av nye. Bedrifter må derfor vedlikeholde mange ulike konfigurasjoner, og vedlikeholde systemer er noe av det mest utfordrende siden det krever så mye tid. Derfor er det viktig å se på løsninger som tar til hensyn til dette. Utvikle abstrakte, high level design tme quality software. Viktig med arkitektur. Problem er at de fleste foretrekker å levere noe i tide enn å lage noe bra.

Trenger en platform for å kunne vedlikeholde TG kontinuerlig, veilede, prioritere, håndtere, refactor.


...............

Most of the future computing systems will be embedde systems. Such systems will integrate both hardware and software components\cite{wolfmadsen-2000}. Embedded software are specialized hardware it runs on, which gives us some constraints when it comes to run-time, like memory usage, processessing power etc. With Internet of Things as a big trend now, these types of software has a big role today. 

One of the problems ES faces is the 

\section{Virtualization of embedded systems}


\section{Configuration management}
Configuration management er et disiplin for styring av av innhold, endringer, status på delt informasjon i et prosjekt. Det omfatter både prosesser og tekniske løsninger for å håndtere endringer og integriteten til prosjektet.F.eks hvis man utgir ny versjon av et produkt, må dokumentasjon også oppdateres. Config Management identifiserer hver komponent, og holder rede på alt som har blitt foreslått og godkjent endring fra dag 1 til slutt.

Software CM er en disiplin for kontrollering av programvaresystemer. Altså kontrollere utviklingen av store og komplekse programvarer. Noen eksempler på SCM er Git-Scm, SVN, RCS, Adele, ClearCase osv. Versjonskontroll er nøkkelen bak SCM. Følgende aspekter er med å definere CM ifølge IEEE:
-	Identification: Struktur av produktet, identifiserer komponenter og deres typer, gjør den unik og tilgjengelig på en måte.
-	Control: Kontrollerer release og og endringer av et produkt i løpet av produktets livsyklus ved å ha diverse kontroller/sjekk som sørger for konsistent produkt via "creation of a baseline product"
-	Status Accounting: Tar opp og rapporterer status til komponenter og evt endringer (forespørsler). Får også  statistikk om produktet.
-	Audit and Review: Validerer det komplette produktet og vedlikeholder konsistensen mellom komponenter ved å sørge for at produktet er en vel-definert kolleksjon av komponenter.

Kan utvides med disse tre definisjonene å:
-	Manufacture: Vedlikeholde konstruksjon og bygge produktet på en optimert måte. 
-	Process management: Passe på organisasjonens personvern, prosedyrer og livssyklus modell.
-	Teamwork: Sjekke arbeidet og passe på et godt samarbeid, og passe på interaksjonen mellom flere brukere og produktet.

Når skal CM brukes? Det varierer. Noen velger å bruke CM system når produktet har gått gjennom utviklingsfasen og er klar for lansering/shipment. Andre velger å putte alt i CM ved oppstart av prosjektet. Begge har sine overheads. Man kan ta et valgt basert på overheads ved en endring. Er det mye manuell arbeid som å fylle ut diverse former, søke om godkjennelse osv vil man ofte plassere programverer under CM etter utvikling. Men hvis en forespørslen om en endring bruker lite tid og innsats fra utviklere, kan man velge å implementere tidlig. I teorien kan CM implementeres i alle stadiger i produktets livssyklus som opprettelse, utvikling, release, levering til kunde, bruk av produkt osv. Men ideelt sett bør et CM ha lite overhead som mulig, slik at software til CM implementeres så tidlig som mulig. Eksisterende CM systemer fokusterer dessverre på en viss fase i livsfasen, så brukere er begrenset av funksjonaliteten.



Ved å velge en robust SCM system gjør det oss mulige til å håndtere store og komplekse filmengder, støtter distribuert utvikling. En riktig kombinasjon av SCM system og "best practices" gjør det mulig for embedded development projects i å progressere raskt og effektivt.


Noen utfordringer med utvikling av embedded systemer er følgende:
-	Complex file sets
o	En embedded system består av flere komponenter, både hardware og software. Dette gjør systemet komplekst siden et slikt system kan ha mange varierte komponenter. Systemer kan også ha ulike varianter av komponenter til en spesifikk platform slik at man kan selge t produkt ved å endre ltit på krav. Å håndtere disse variantene er en stor utfordring. En annen utfordring er at produkter krever en korrekt versjon av en komponent. Å sørge for at korrelasjonen mellom hver komponent og deres avhengige filer er vedlikeholdt er en utfordring det å.
-	Distributed teams
o	Komponenter kan utvikles i ulike steder i vår verden. Samtidig kan to teams to forskjjelige steder jobbe med samme komponent, spesielt når noe outsources. Slikt samarbeid krever at utviklere har adgang til hver andres arbeid. Utfordringen er at utviklere som jobber i hvert sitt sted (geografisk) holder seg synkronisert.
-	Management and versioning of intellectual property
o	Siden embedded systemer ofte tar I bruk tredje-parts teknologier, er det viktig at de utviklerne bak disse teknologiene oppdater og vedlikeholder arbeidet sitt. Disse oppdateringene må også være sporbare slik at prosjektet inneholder riktig, kompatibel og stabil versjon av teknologien. Utfordringen er å tillate disse utviklerne å sjekke inn contributions og spore endringer i det man har kontribuert. Velge man noe open source ervel dette ikke et problem?




Når det kommer til teknisk gjeld er det ikke alltid personen som har utviklet noe som tar ansvar, men kan en annen kan ta seg av den gjelda. Mange utviklere vedlikeholder ikke sin egen kode. Mange selskaper har også regler om at når et software er ferdig utviklet av de "beste" til å bli vedlikeholdt av de nest beste, som ofte kan få mindre betalt men har mye mer arbeid å gjøre. Ingen i din organisasjonen viser interesse for det, er brukerne som må betale for gjelda. Utviklere er belønnet for hvor raskt de implementerer enn langsiktig vedlikehold og kan ha fått seg et nytt prosjekt før gjelda er betalt. Få systemer har TODO eller FIXME kommentarer i kildekoden. 

Til forskjell fra finansiell gjeld kan teknisk gjeld aldri betales tilbake i sitt fulle. Å betale TG kommer i en form av hvor lang tid det tar å fikse koden/problemet. Men det er ikke lett å vise hvilke gjeld som har høyest kost. Er interessen lavere enn hva det koster, er det ikke vits i å betale tilbake. Eksempel: Man har et system som trenger en oppgradering som kan koste 1 million. Man tar valget i å ikke oppgradere, og satse på at systemet fungerer. Det gjør det ikke, systemet går ned og firmaet taper felre millioner på å reparere systemet. Her kunne man spart penger på å oppgradere.


\subsection{Software reuse}
Software reuse is about using existing software artifacts, or knowledge, to create new software, rather than building it from scratch. Software reuse is a key method for improving software quality. 


\subsection{Software development life cycles} % (fold)
Weathered j-pop tube paranoid systema marketing sprawl warehouse boy receding corrupted footage DIY order-flow Chiba. Rain narrative construct shoes kanji faded bicycle denim girl. Gang car city sentient silent drugs dolphin man. Garage lights face forwards motion monofilament stimulate table sensory network disposable cartel systema sign katana. Man apophenia pre-A.I. papier-mache pistol range-rover rain computer receding hacker saturation point. 

\subsection{Software evolution} % (fold)
Soul-delay shoes neural wonton soup hotdog BASE jump knife towards A.I. engine concrete katana network bridge cartel artisanal. Rain girl shanty town advert into uplink pre-convenience store camera singularity DIY marketing Tokyo papier-mache j-pop nodal point. Fluidity hotdog camera cyber-assassin bomb 8-bit denim concrete shoes. A.I. weathered nodality corporation voodoo god industrial grade soul-delay jeans 8-bit. 

\subsection{Deployment} % (fold)
Systema beef noodles rifle papier-mache wristwatch Legba futurity city meta-Kowloon. Tattoo film otaku carbon sprawl bomb vehicle boat refrigerator San Francisco woman grenade meta-tube neural digital. Courier uplink chrome convenience store marketing Chiba refrigerator smart-modem 8-bit. Post-skyscraper claymore mine towards Kowloon hotdog savant rebar camera. Dead jeans weathered neural carbon paranoid gang A.I. nano-office papier-mache youtube 8-bit corrupted. Weathered paranoid cardboard tiger-team 3D-printed kanji man savant faded footage rebar. Man smart-denim convenience store pre-towards kanji nano-hacker soul-delay woman dome physical. 






