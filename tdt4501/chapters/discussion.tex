% !TEX root = ..\main.tex
\chapter{Discussion}
This chapter discusses the results.

% Svare på forskingsspørsmål
%In Section X.X, the research questions were outlined. Below is a summary of how the research questions have been addressed. Several sources of technical debt were identified.
% Knowledge might be a problem too (I4, source. Modul greia).

% Relasjoner, likheter, ulikheter

% Mappe opp mot RQ

\section{Summary of the findings}

The most common cause of technial debt that was mentioned was time issues. Two of the interviewees mentioned that technical debt is incurred due to lack of time. This is something we recognize from Cunninghams artcile. Neitherless, one of the interviewees mentions that they compensate the customer with a product of high-level quality. The other interviewer calculates the risk on their debt. Highest risk issues are prioritized first. Using third-party solutions is considered as a source of technical debt. Such solutions might be supported for a period. After that, the choice is to replace, or adjust the code.

The research also reveals that technical debt is connected with many different aspects in the software development life cycle. An architectural solution in one of the projects resulted in bad outcomes. This is known as architectural debt. It is impotant to notice that shortcuts was not taken to meet deadline, but the solution itself was not optimal to solve the problem. McConnell defines this as unintentional debt. Software architecture is one of the artefacts that is hard to change, and therefore the debt becomes much higher. Lack of tests, and shortcuts taken is tests were also revealed as a source of technical debt. This is known as test debt. Another source of technical debt that was identified is the combination of old equipment and old code. The problem with old equipment is that they are vulrnearble.

The research revealed that technical debt is incurred intentionally in order to meet a dealine. This happens with or without a plan on how to pay it back. The short-term benefits are customer satisfaction, and that the company gains competitive advantage. However, some of the long-term effects is that the complexity in code keeps increasing, extra workign hours, error and bugs. I1 mentipned that they often has code nights. However, sometimes technical debt is incurred unintentionally. This can be compared with McConnels definition. Technical debt can be unvinsible as well, related to the quadrant.

One of the interviewees also mentioned that they have multiple projects. It seems like they have problems on balancing technical debt and developing new functionalities. Many of their systems does not get updated, such systems creates security breaches and might have troubles with scalability later. Think about it.

It also revealed that the technical communication gap between the team and the management is high. From the interviwerrs perspective, the management remains largely unaware of technical debtm and that they cannot see the value behind technical management. Therefore it is hard to convince the management. 

It is interesting to see that the way the embedded area and the software area handles technical debt is different. The studies revealed that technical debt in the software projects are much higher than the embedded system projects. One of the interviewees mentioned that their project used lots of out-of-date third-party components. After changing to open source solutions, they managed to reduce technical debt drastically. This reveals that technical debt is taken much more seriously by the embedded system area. This is due to that their solutions cannot contain any errors. However, both of the embedded systems projects has some technical debt in their project, but it is something they are able to manage. The most important thing is that the product is stable, and the quality is good. HOwever, if we compare the embedded systems, we can see that the reason that the last one is able to manage technical debt is because they develop their own frameworks. Using open-source solutions, or developing frameworks, gives lots of benefits. One thing is that the frameworks can be reused in other projects. TDD is also used in both of these projects, it is important to test the product thoroughly before it is put on production. I4 further explains that their technical debt is more on the frameworks than the product itself. Some lack of tests as well.


\section{Research Questions}

\subsubsection{RQ-1: What practices and tools for managing technical debt? How are they used?}
The first research questions focuses on strategies of managing and reducing technical debt. Neither of the interviewees had any specific management strategy for technical debt. Neitherless, all of the interviewees were using some practices to manage technical debt. A common solution to handle technical debt issues that was expressed by the interviewees is to make sure that the developers are aware of the technical debt issues. If the company knows that there is some lack of tests, or that a program has architectural issues, the debt would be less significant. Development team used a backlog to collect technical debt issues and their risk along with new functionalities to be implemented. Technical debt issues can be taken into account when planning feaure implementation, which lessens the impact of the debt. Configuration management systems such as Jira, Git, is used for creating change requests. Issue trackers in version control systems are also used for both functionalities and technical debt. Change requests are often made by the developers, and the risk behind every change is calculated by the management. 

Practices such as refactoring, bug fixing days, reengineering were also identificed in this research even though the organizations has any strategy for technical debt management. Similar practices has been suggested in other studies (SITER). It is believed that these practices can reduce and prevent the amount of technical debt and also increase the overall quality of the product.

Another method was the use of test-driven development. One of the interviewees used this method. Writing tests helps identifying and dealing with bugs.


\subsubsection{RQ-2: What are the most significant sources of technical debt?}
The second research questions looks at some of the reasons the organizations incurred technical debt in the software development process. The results suggest that technical debt is not caused by a specific reason. Lack of time in the development phase was identified as the primary reason for technical debt in a software project. Several authors has identified that time is one of the reasons of going into TD (SITER). 

Archnitectural choice was also mentioned as a source of technical debt. It is not related to shortcuts taken just to finish the architecutre, but more than the architecture choice that was taken is not necessary the best choice. The whole architecture is a debt. Such issues are hard to deal with.

Lack of tests was mentioned as a source.


\subsubsection{RQ-3: When should technical debt be paid?}
- When risk is high and it is causing big problems.
- Some debt doesnt need to be paid off

\subsubsection{RQ-4: Who is responsible for deciding whether to incur, or pay off technical debt?}

 Management is mostly responsible here. Develop team might incur small amount of technical debt. Pay off technical debt depends on the problem. Developers decides sometimes if they got time, but most of the times it is management and project leader.

\section{Study limitations}

\subsection{Internal validity}


\subsection{External validity}
Only 4 people were able to participate in the interviews. 

\subsection{Construct validity}

