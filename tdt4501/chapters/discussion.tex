% !TEX encoding = UTF-8 Unicode
% !TEX root = ..\main.tex
% !TEX spellcheck = en-US
\chapter{Discussion}
\label{chap:discussion}
The purpose of this study was to do a broader review of how technical debt in embedded systems.
% Svare på forskingsspørsmål
%In Section X.X, the research questions were outlined. Below is a summary of how the research questions have been addressed. Several sources of technical debt were identified.




% Knowledge might be a problem too (I4, source. Modul greia).

\section{Definitions of Technical Debt}
The definitions of technical debt given by the participants have many similarities with the literature. technical debt was defined in terms of code refactoring, which matches Cunningham's definition of technical debt\cite{p29-cunningham}. Furthermore, technical debt was also defined in terms of using legacy solutions. McConnell classifies technical debt as intentional and unintentional debt\cite{url-mcconnell}. By looking at the material from the interviews, it is possible to see that technical debt was mostly associated with intentional debt. One example was that developers experimented different solutions to a problem by hard-coding. Solving problems by hard-coding results in short-term benefits, but refactoring is needed at a later point. Moreover, the interview data revealed that technical debt is not always bad. In some situations, intentional debt brought success in terms of reaching goals, such as delivering products quickly to customers. Additionally, the interview data revealed some situations where technical debt was incurred unintentionally. Such decisions afftected the overall software quality.

Workaround is not something that has been mentioned by the literature, but it is enough to describe as some of the symptoms of technical debt. Workarounds are related to code design which is not obvious and easy to understand. It can be used to bypass difficult points in the code which may result in technical debt. Moreover, workarounds can be taken in other phases in software development, which makes the concept much closer to technical debt. As described in one of the interviews, workarounds were often taken in legacy systems.

\section{Causes of Technical Debt}
%closing the gap between technical and business.

% FLYTT P1 and P2 revealed that requirements from the management or customers are the dominant factor in determining if the available resources should be used to address technical debt, while deadlines are the dominant factor. 

Time pressure was revealed as a common cause for technical debt accumulation. Issues with time has been present in multiple articles as a reason for incurring technical debt\cite{p29-cunningham,foser076-brown,zazworka2011investigating,p8-codabux,p50-allman,lim-taksande}. Moreover, Ebert et al.\cite{ebert2009embedded} states that companies compress schedules to a point that makes engineers compromise design-time qualities to run-time qualities. Time issues ultimately comes from business realities that needs to be met based on customer needs and market situations. 

Time issues may lead to communication problems within the team, thus incurring technical debt intentionally. Fowler defines this situation as reckless and deliberate technical debt\cite{url-fowler}. This could be due to stress or lack of knowledge. As the project moves along, the debt will eventually surface, and when it does it will suddenly need to be repaid. Nevertheless, the findings revealed that time pressure was manageable.

Furthermore, the research identified that technical debt is connected with many different aspects in the software development life cycle. Table \ref{tab:subcategories} lists the subcategories of technical debt. We clearly see that there many similarities between. When the problem of technical debt is divided into subcategories, it is easier to become aware of the problems of technical debt. An architectural solution resulted in bad outcomes. This is very similar to architectural debt, and unintentional debt. Software architecture is one of the artifacts that is hard to change, and therefore the debt becomes much higher. Code refactoring is needed before new functionalities could be added. This is very similar to code debt\cite{li2015systematic}. The interviews also revealed some lack of tests, and that shortcuts were also taken in test code. This is known as test debt\cite{li2015systematic}. Use of older technologies and framework are related to infrastructure debt\cite{li2015systematic}. The findings also revealed lack of documentation in some of the third-party solutions, known as documentation debt\cite{li2015systematic}. By dividing technical debt problems into categories based on where and why they occur, the actual problem would become more specific and this easier to grasp. This makes it easier to allocate respectability for making sure that debt is correctly managed.

Poor choices of technologies and third-party solutions from external suppliers can be considered as a debt generating activity. The participants expressed how technical debt accumulates when suppliers are going out of business, or are unable to support the product with updates. It is very rare for companies to select a poor implementation intentionally, to get short term benefits. However, poor technology choices will eventually surface as the system grows and its bottlenecks are discovered. It is something that cannot be planned for, and requires time and effort to fix one discovered. When choosing technologies or third-party solutions from external suppliers, it is important that developers has the competence to use the technologies, or that enough documentation follows the solutions that is bought from external suppliers. One of the participants mentioned that lack of documentation led to major workarounds later on.

\section{Incurring Technical Debt}
Observations from the interviews indicated that there is a difference on incurring technical debt between developers and management. Project managers were more likely to incur technical debt, because the realization of meeting their business goals, without any plans on paying it back. It also revealed that if a product is working, it should not be touched. This indicates that the management does not trust the developers. Moreover, from the developer perspective, the management remains largely unaware of technical debt, and they cannot see the value behind technical debt management. It makes it hard to convince the management. Overall, the technical communication gap between the business and development departments is high. Klinger et al.\cite{p35-klinger} found similar causes in their IBM study. Based on this, we believe that larger companies has more challenges dealing with technical debt issues due to complex communication structure. We believe that management with no knowledge about technical debt is responsible for technical debt accumulation in this research. Development teams may incur small amount of technical debt, but not more than what they are able to handle.

Several studies argue that the short-term effect of time-to-market is a good thing about technical debt\cite{lim-taksande,p247-siebra}. The interviews revealed similar situations, where technical debt were used to deliver a solution faster to the customer, resulting in customer satisfaction. We believe that the customer and management will demand more over time, causing accumulation of technical debt, leaving them unhandled. Moreover, the long-term effects of technical debt tends to have more negative effects\cite{lim-taksande,p247-siebra,buschmann2011pay}. The findings revealed that technical debt in the long-term started to generate extra working hours, performance issues, scalability errors, and system failures. By taking technical debt, things might turn into a problem later if they are not paid back. This might be a reason for why the amount of technical debt has doubled since 2010\cite{gartner2010}. Business people often thinks that technical debt is something that can be incurred to reach a deadline, and just fix it later. This is very similar to reckless and deliberate debt\cite{url-fowler}.

The findings also pointed out difficulties with ongoing projects, where each project has low amount of available resources. It seems like the participants have problems on balancing technical debt in parallel with development of new functionality. Many of their systems does not get updated, which creates problems in the long-term.



% Krutchen claim that "Most authors agree that the major cause of technical debt is schedule pressure", although other issues can come into play, like lack of knowledge, carelessness, basic incompetence, lack of educiation.

% Klinger claim that debt is result of stakeholders that lack effective means to communicate.



\section{Management of Technical Debt}
Although technical debt is not considered in different projects, it is still managed thorough the development and evolution by using different practices. One of the practices that was mentioned is very similar to managing risks. Like risk management, technical debt management is a balancing act that aims to achieve a level of good quality while mitigating its failures. In particular, the participants had to not only evaluate the technical implications before making a choice, but also the impact on the delivered business value. For example, such trade-offs included release of the product in time to capture the market share. This requires involvement of all the stakeholders. The impact and consequence of technical debt can be used to convince the stakeholders to agree upon the same strategy for managing technical debt. A similar strategy has been defined in the literature\cite{lim-taksande,Theodoropoulos:2011:TDS:1985362.1985373}. The author believe that closing the communication gap between technical and non-technical stakeholders is important to increase the visibility of technical debt. 

It was evident from the interviews that communication within the team played a big part in achieving higher software quality. A common practice to handle technical debt issues expressed by the participants, was to make sure that the developers are aware of the technical debt issues. For example, if a company knows that there are some lack of tests, or that a system has performance issues, the debt would be less significant. The different development team used a backlog to collect technical debt issues and their risk in parallel with new functionalities to be implemented. technical debt issues can be taken into account when planning feature implementation, which lessens the impact of the debt. Similar strategy has been suggested by Krutchen et. al\cite{krutchen}. A portfolio management strategy has been proposed by Guo et. al\cite{guo2011tracking}, where technical debt is stored in a backlog and development team can use that for management of technical debt. This backlog strategy may be beneficial in the long-run when older technical debt is traceable, instead of of forgotten.

The use of test-driven development was identified as a way to manage technical debt thorough the development. Using test-driven development makes sure that bugs are discovered in the software. This improves the quality of the software. Test-driven development has been suggested by Ebert et. al\cite{ebert2009embedded} as a practice to improve the software quality.

A common approach to keep the debt from growing over time is to conduct refactoring and re-engineering. Both refactoring and re-engineer has mentioned by the interview participants, and the literature\cite{p8-codabux}. Code refactoring was applied by developers in parallel with development. Sometimes, developers spent evenings on code refactoring. The author believes that refactoring keeps the software quality stable, and mitigates technical debt issues. 

Besides the use of practices that has been mentioned, it would be preferable to set apart a certain amount of time during each iteration, to address and manage technical debt. Similar practice has been suggested by Codabux et. al\cite{p8-codabux}. These measures may not have a big impact for the first few iterations, but one would be able to see improvements in quality and productivity over time. 


\section{Research Questions}
It is interesting to see that the way embedded system developers and traditional software developers handles technical debt. This research revealed that technical debt in traditional software projects are much higher than the embedded system projects. Despite the re-engineered product by one of the participants using open-source technologies, they were at least aware of the amount of accumulated technical debt in their old solution. Fowler defines this as inadvertent/prudent debt\cite{url-fowler}. Additionally, this indicates that technical debt and software quality is taken much more seriously by the embedded system industry. Research has revealed that embedded software cannot contain any errors\cite{ebert2009embedded,trienekens2010quality,pretschner2007software}. 

Despite the fact that embedded system developers are more serious addressing technical debt, this research also revealed that embedded system projects have some technical debt in their project. However, the amount of technical debt accumulated in the various products are manageable. The most important thing is that the product is stable, and the quality is good. By using open-source solutions, or developing own frameworks gives lots of benefits. Frameworks that has been developed in-house can be reused by other projects. However, developing own solutions poses risks. One of the participants pointed out that their testing framework were out-of-date. As mentioned in Section \ref{sec:2-SR}, reusing out-of-date solutions is the root cause of technical debt accumulation.

The research questions that were outlined in Chapter 1 are following:
\begin{itemize}
	\item \textbf{RQ1}: What practices and tools exists for managing technical debt? How are they used?
	\item \textbf{RQ2}: What are the most significant sources of technical debt?
	\item \textbf{RQ3}: When should a technical debt be paid?
	\item \textbf{RQ4}: Who is responsible for deciding whether to incur, or pay off technical debt?
\end{itemize}

Table \ref{tab:rqanseer} contains a summary of the results of this research. Although, the results did not specify any specific approach for managing technical debt, this research were able to identify some practices for reducing technical debt. Moreover, this research were able to identify different sources for technical debt. This study was also able to identify when technical debt should be paid off, and the people responsible for taking such decisions.

\begin{table}[!ht]
\centering
\caption{Research Questions and Their Findings}
\label{tab:rqanseer}
\begin{tabular}{|l|p{12cm}|}

\hline

\textbf{RQ1} & Test-driven development: One of the participants used this methodology to address technical debt thorough development. \\ \hline
\textbf{RQ2} & Lack of time given for development was identified as the primary source for incurring technical debt. Time pressure was often caused by business decisions. Other sources of technical debt that were identified are lack of tests, architectural choice, and technology choices. \\ \hline
\textbf{RQ3} & The risk of technical debt items determined whether companies should invest resources in paying down the debt. If the risk was high, an evaulation of the risk was made. Additionally, the results also outlined that some debts does not need to be paid off.\\ \hline
\textbf{RQ4} & A combination of management and the customer were mostly responsible to incur, or pay off technical debt. The risk behind debt was measured before deciding whether to pay off or not. This research has also revealed situations where developers incurred technical debt, both intentionally and unintentionally. \\ \hline
\end{tabular}
\end{table}

%\subsubsection{What practices and tools for managing technical debt? How are they used?}"
%- Test-driven development
%- Overview of the technologies, systems, frameworks being used, along with their versions.
%- Refactoring and re-engineering
%- Backlog and issue tracker
%- Communication structure between business management and development team


%\subsubsection{What are the most significant sources of technical debt?}
%- Lack of time given for development
%- Pressure to the development team
%- Business decisions
%- Architectural choice
%- Lack of tests

%\subsubsection{When should a technical debt be paid?}"
%- When risk is high and it is causing big problems.
%- Some debt does not need to be paid off


%\subsubsection{Who is responsible for deciding whether to incur, or pay off technical debt?}"
%- Management, customers, and developers. Depends on risk and situation.


\section{Threats to Validity}
\label{sec:threat}
Validity is related to how much the results can be trusted\cite{Wohlin:2000:ESE:330775}. Wohlin et al.\cite{Wohlin:2000:ESE:330775} states that adequate validity refers to that the results should be valid for the population of interest. First of all, the results should not be valid for the population from which the sample is drawn. Secondly, it may be of interest to generalize the results to a broader population. Wohlin et al.\cite{Wohlin:2000:ESE:330775} describes four types of validity threats; internal, external, construct, and conclusion validity.

\subsection{Internal validity}
Internal validity concerns about the "right data"\cite{Wohlin:2000:ESE:330775}. If the outcome is caused by the treatment and not by other factors not measured, we can conclude that the result has internal validity. The relevant threats for this research are:
\begin{itemize}
\item Maturation: The participants may be affected negatively during the interview. They could be tired, or not motivated to answer some questions. 
\item Instrumentation: This effect is caused if the artifacts of the interview is badly designed. We do not think that the interviews were badly designed, and helped the participants with additional information if needed.
\end{itemize}


\subsection{External Validity}
External validity is the degree to which the results of an experiment can be generalized outside the experiment setting. External validity is affected by the chosen experiment design, the objects in the experiment, and the subjects chosen. There are three main risks: people, place, and time. This research is concerned about people. Only 4 people were able to participate in the interviews, and they are not possibly not representative for the larger population. Additionally, only two of the participants were from the embedded system industry. We cannot conclude that the results from this research reflects a more general trend. More studies will be necessary to generalize the results.

\subsection{Construct Validity}
Threats to construct validity refer to the extent to which the experiment setting actually reflects the construct under study. Questions that were asked during the interviews may have been misunderstood because they were improperly phased. The participants may therefore answer something else. After conducting and analyzing the interviews, we realized that some questions could have been better phrased to gain deeper information about the topic. The interview guide was reviewed by the supervisor, but we did not test the interview guide before the interviews.

\subsection{Conclusion Validity}
Conclusion validity concerns the ability to draw correct contusions from the relationship between the treatment and the outcome\cite{Wohlin:2000:ESE:330775}. Because we only had 4 participants, the statistical power is very low. This is something we are aware of, and therefore more thorough studies need to be performed to confirm if our results have more general applicability. Additionally, the authors lack of experience creating the interview guide may affect the conclusion as well.



