\chapter{Sammendrag artikler}
\section{Sammendrag oppskrift}
- Description
- Goal (find out something, maybe give organization something)
- Research method
- Findings

\section{P8-Codabux}
Description: One of the root failures for software development is use of the sequential design processes for building complex, intensive systems. In sequential design processes, the remaining software development activities is based on the initial requirements. So this type, eg. Waterfall, is not appropriate to use when business needs to change business needs and technology rapidly. This is where agile development comes in. One primary benefit is the quick release of functionality. However, the focus can change from implementing new functionality rather than design, good programming practice and test coverage.

Goal: Codabux goal is to research agile development focusing on TD in an industrial context. Gain software developers insights on agile adptopn and how technical debt affects the development process. The objective is to identify the main challenges and best practices with agile adoption and the management of technical debt.

Research method: 3 on-site process evaluations with their industrial partner since their initial adpotion of the Agile methods., in the form of an interview

Findings: After conducting their research methology, they got two definitions of technical debt, infrastructure and automation debt. Infrastructure is the work that improves team's processes and ability to produce a quality product. Example of this is refactoring, repackaging, developing unit tests. Automation is the work related to supporing continous integration and faster development cycles. Make sure that previuous tests are running.

There are indications that technical debt is influenced by lack of time needed to properly design and code new features. The participants relates technical debt mostly to design, testing and defects debts.

One of the engineers doesn't know the "balance of the credit card, and keep on charging it.". Management decides when enough debt has incurred, and the management is influenced by the customer needs.

Some debt stay in the backlog if it's not a priority because the focus is on new features.

\section{P35-Klinger}
Description: Klinger describes an industrial case study on technical debt carried out at IBM.

Goal: They want to find out how technical debt is analogous to using leverage for long-term investment (the extent to whicn borrowed money is being used in an investment or a business). For example, to meet the delivery date of a product, techincal shortcuts might be necessary. However, meeting the delivery date might ensure that the product thrives and grows in marketplace, something that might be en jeopardy if delivery date is missed. 

Research method: Ethnographic study in the form of a series of interviews involving four technical architects with different backgrounds. Goal here is to examine how the decisions to incur debt were taken and the extent to which the debt provided leverage.

Findings: The company fails to assess the impact of intentionally incurring debt on projects. Decisions regarding technical debt were rarely quantified, and organizational gaps among the busness, operational and technical stakeholders contribute to incurring debts.




\section{p73-bennet}
Description: Software maintenance and evolution are characterized by their cost and implementation speed. But they are inevitable activities, all software that is successfull can be changed and improved.

Goal: Describe a landscape for research in software maintenance and evolution for the next ten years. This research has a goal to improve the implementaitn speed and accuracy while reducing the costs. This is done by identifying key problems, promising solution strategies and topics of importance.

Research method: 

Findings: 