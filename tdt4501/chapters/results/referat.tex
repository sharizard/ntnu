\section{FørsteMøte}
- Litteratursøk
- rapportmal
- undersøkelser
- forstudie
- verktøystøtte




\section{AndreMøte}
Prestudy
- Software reuse
- Config management
- Software development cycles
- Software evolution
- Deployment, embedded
- hvordan SW vedlikeholdes
- TG evolution SW
- config management
- smidig
- TG mer mot arkitektur, hvordan påvirker det utvikling (DB, IoT, web)

Arkitektur og design nå, hvordan endringer påvirker fremtidige valg

Identifisere gjeld via undersøkelser, analsyser, utvikle verktøy, modeller, metoder for dette. Kartlegg virksomhet

Tar ofte alt det over i bruk uten å tenkte på TG. TG kan allerede oppstå i kravfasen. 

Embedded system går ut på å sette sammen små komponenter til noe større. Vedlikehold av en komponent, oppdaterin,g kan påvirke andre.

Deprecated software ikke lett å vedlikeholde. 

Continious integration


\section{TredjeMøte}
- Økonomisk og organisatorisk gjeld. Hvorfor oppstår dette? Lønner det seg å betale?

- Wrappe ny funksjonalitet rundt det gamle. Vi betaler da ikke TG, men løser det.

- Embedded system, se på enterprise sammenheng. Probabilitets modellering.

- IoT, protokoll og sikkerhetsproblematikk siden det ikke er utviklet til nettbruk.

- Abstraksjonsmetode på prog språk, PLS -> embedded. Trenger emulatorer for å unngå problemer. Kjøpe inn ny HW og emulere. Forholde seg til oppgraderinger av språk, systemer osv.


\section{FjerdeMøte}
- Outsourcing av IT-portefølje. Større leveranderer kan ofte føre til TG?

- Økonomisk perspektiv -> ikke stor fokus på det men bør nevnes. Ha fokus på prosess (software development process) og forvaltning.

- Pådrar seg TG når man begynner på noe, men oppstår når utvikling skjer. Kan være lite penger til utvikling/ressurser f.eks.

- Smidig -> ikke krav til maintenance/vedlikehold. Ingen forvaltningsperspektiv tidlig i kravspek fasen. Bruker/kunde har heller ingen formeninger om forvaltning.

- ENterprise arch -> TG. Ikke investert nok, dårlige arbeidsprosesser f.eks.

- Deprecated, kompabilitet på OS nivå, progspråk nivå osv.