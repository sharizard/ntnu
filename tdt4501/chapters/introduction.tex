\chapter{Introduction}

\section{Motivation and goals}




Today, hardware and software based systems, further known as embedded systems, are growing rapidly. Ranging from microprosessors in cellphones to sensors in cars and surveillance systems, embedded computing is becoming a big part of our lives. We can see that most future computing systems will be embedded systems\cite{wolfmadsen-2000}. With the new era of Internet of Things, more and more embedded systems are getting a networked interconnection. This leads to a distributed network of devices communicating with other devices as well as human beings. Gartner has estimated that in 2020, 25 billion connected "things" will be in use\cite{gartner}. To provide more functionality, multiple components are combined together with embedded systems. However, as the complexity of embedded system increases, the ability to maintain the quality of such systems becomes more difficult. Combination of multiple components leads to higher costs of verifying additional software and many fails to test the product properly and deliver a reliable product. Companies must often recall their products. If they could catch these software defects earlier in teh system design process, they would have saved a lot of money. Embedded systems also has long lifetime and its important to find out how to make decisisions so future maintenance and evolution has low cost as possible. Technical debt is a big factor in embedded systems as developers might not be available years after implementation.

Technical debt is a rising problem. It is estimated that the total debt in 2015 will be around 1 billion dollars. That is the double of the amount technical debt in 2010.


\section{Research questions}
The main question for this project is: 
\begin{itemize}
	\item Hva er teknisk gjeld? Klarer vi å lage en definisjon til metaforen?
	\item Hva blir teknisk gjeld tolget som?
	\item Hva slags synonymer finnes til TG?
	\item Hva slags typer TG finnes? 
	\item Hvordan skiller teknisk gjeld seg ut fra finansiell gjeld?

	\item Hva slags teknikker finnes til håndtering av TG, og hvordan blir disse brukt?
	\item Hvordan er TG relatert til evolusjon og maintenance?

	\item Hvordan prioriteres hva som skal nedbetales først?




	\item Current state of TD in the systems?
	\item How much do you focus on technical debt? I hvilken grad fokuserer virksomheten på dette?
	\item How much time is spent on reducing technical debt?
	\item Pådra gjeld med vilje? Fikses det?
	\item How much focus there is on maintenance and evolution?
	\item When a debt should be refactored?
	\item Which debts are easy to fix which gives software quality boost?
	\item Which debts are most difficult to address, which debts are hard to fix which doesnt give much software quality boost?
	\item What kind of methologies are used to track/prioritize technical debt
\end{itemize}


\section{Research method}
This research will consist of a sequence of activities. I've chosen to follow the model of the research process defined in the book "Researching Information Systems Something". This model says that for each research question, one strategy is needed with one or more data generation methods. A dataanalysis will be done after generation in form of quantitative or qualitative in order to draw a conclusion. This is to answer the research questions.

First of all, a litterature review will be carried out in order to get familiar with the area of study. Following the litterature review, one or more research questions will be defined. A strategy will be choosed, in this case a case study/survey
. I will use the interview form as data generation

\section{Project structure}
The rest of this report will be structured as followed:

- Chapter 2: State-of-art of technical debt, embedded systems, software evolution, software maintenance, software development life cycles, software reuse, configuration management, security.

- Chapter 3: This chapter will present our reseach context and research method. 

- Chapter 4: Will present the result of this project.

- Chapter 5: Evaluate and discusses contributions and the results to this project

- Chapter 6: Concludes the report and provides points to future work. 
