% !TEX root = ..\main.tex
\chapter{Research Method}
This chapters presents the main empirical strategies used in this research.

\section{Empirical Strategies}
Empirical studies follows two types of research paradigms; the qualitative, and the quantitative paradigm\cite{Wohlin:2000:ESE:330775}. Qualitative research is concerned with studying objects in their natural setting\cite{Wohlin:2000:ESE:330775}. Its data include non-numeric data found in sources as interview tapes, documents, or developers' models\cite{Oates:2006:RIS:1202299}. Quantitative reseach is concerned with quantifing a relationship or to compare two or more groups\cite{Wohlin:2000:ESE:330775}. It is based on collecting numerical data\cite{Oates:2006:RIS:1202299}.

Oates\cite{Oates:2006:RIS:1202299} presents six different research strategies; survey, design and creation, case study, experimentation, action research, and ethnography. 

\textit{\textbf{Survey}} focuses on collection data from a sample of individuals through their responses to questions. The primary means of gathering qualitative or quantitative data are interviews or questionnaries. The results are then analyzed using patterns to derive descriptive, explorative and explanatory conclusions. \textit{\textbf{Design and creation}} focuses on developing new IT products, or artefacts. It can be a computer-based system, new model, or a new method. \textit{\textbf{Case study}} focuses on monitoring one single 'thing'; an organization, a project, an informaton system, or a software developer. The goal is to obtain rich, and detailed data. \textit{\textbf{Experimentation}} are normally done in laboratory environment, which provides a high level of control. The goal is to investigate cause and effect relationships, testing hypotheses, and to prove or disprove the link between a factor and an observed outcome. \textit{\textbf{Action research}} focuses on solving a real-world problem while reflecting on what happened or what was learnt. \textit{\textbf{Ethnography}} is used to understand culture and ways of seeing of a particular group of people. The researcher spends time in the field by participating rather than observing.

\subsection{Choice of Methods}
Out of the six research strategies presented by Oates\cite{Oates:2006:RIS:1202299}, survey was chosen as a strategy for this research. As mentioned before, questionnaires and interviews are two common means for data collection\cite{Wohlin:2000:ESE:330775}. The method chosen is usually based on type of results researches are looking for. There are three types of interviews\cite{Oates:2006:RIS:1202299}:
\begin{itemize}
	\item Structured interviews: The use of pre-determined, standardized, identical questions for every interviewee. Structured interviews are very similar to questionnaires.
	\item Semi-structured interviews: A list of themes and questions is prepared. However, the interviewer has the possibility to ask additional questions that is not included in the list.
	\item Unstructured interviews: The topic is introduced by the interviewer, and the participants talks freely about related events, behaviour, or beliefs. The researcher has less control. 
\end{itemize}

This research will conduct semi-structured interviews as the supporthing research method for qualitative data collections. Interviews where chosen as the method because it was the best trade-off between usage of time, and its ability to provide detailed textual description for the research questions. Most of the questions were formulated in an open fashion which allowed the interviewee to speak with more detail on the issue with their experiences and viewpoints\cite{Oates:2006:RIS:1202299,Wohlin:2000:ESE:330775}, hence give a qualitative and flexible answer. 

\subsection{Data collection}
The research was performed by collecting data through a series of interviews with companies in the electronics and software business. The interviews focused around how companies encountered technical debt, how they are addressing it, what they are doing to handle it through the development and software evolution. In total, four interviews were conducted with four different companies, two of them working with embedded system development, and the last two working with traditional software business.

The companies to interview were chosen by the author and the supervisor of this project. The interviews were mainly conducted in Trondheim, because the companies could easily be interviewed in person. A proper request for participation for an interview were sent out to companies through e-mail. The participant were chosen by the company, but the type of person needed for the interview were given by the author.

The interviews were conducted in a semistructued manner. The benefit of using a semistructured approach is that the interviewer are able to change the order of questions depending on the flow of the conversation, and to probe deeper into a subject by asking additional questions the interviewer had not prepared for\cite{Oates:2006:RIS:1202299}. 

The interview questions were created by the author with the assistance of the supervisor. These questions were used as a guideline to get an overview over the subjects to be asked, and appropiate follow-up questions were asked to gain in-depth understanding about the subject. The inteviewer took notes while the interview was in progress. 

A certain degree of structure in the interviews also provides a basis for comparing the interviews afterwards. Covering the same main topics makes it possible to extract common trends and differences in the answers. After the interviews were conducted, it analyzed by arranging the interview data in tables containing the most relevant characteristics, and conclusions were drawn from analyzing it.



