% !TEX encoding = UTF-8 Unicode
% !TEX root = ..\main.tex
% !TEX spellcheck = en-US



% viktig å «konkludere» med at utvalget av respondenter er lav, så resultatet kan ikke generaliseres, men er viktige observasjoner.

\chapter{Conclusion}
This research presents the results of empirical studies that has investigated technical debt in embedded systems. This study has revealed that technical debt is something that organizations are unable to avoid during software development projects. However, technical debt is not always a bad thing to take. Organizations can use it as a powerful tool to reach their customers faster and gain edge over the competition in the market. However, if technical debt is not paid back in time, it may generate economic consequences and quality issues to the software. It is necessary for organizations to create a strategy plan that includes practices and tools that decrease technical debt.

Empirical research has been performed to verify theories. This study is mainly used to gain an understanding about technical debt accumulation in embedded systems with the goal to find the best practices for managing technical debt. This study reveals that the way traditional software developers and embedded system developer handle technical debt is different. The reason is that embedded system developers needs to deal with multiple constraints. This research also revealed that the awareness of developers and managers regarding technical debt is an important factor. When they are aware of the technical debt, it is less dangerous as it can be accounted for when planning for the future. Unknown and hidden debt is more dangerous. Having the ability to predict the long-term business outcomes of short-term technical decisions would help the software practitioners to choose the right kinds of technical debt to incur.

As mentioned in Chapter \ref{chap:discussion}, Section \ref{sec:threat}, the results cannot be generalized, since the amount of respondents are low. However, the observations from this study are very important for future research. There are still large amount of future work to be done in this subject area to develop strategies and tools that can help teams successfully managing their debt. 

\section{Future Work}
The future work will consist of the upcoming Master's Thesis, which will have a limit of 20 weeks. There are still plenty of room to look deeper into a more narrow field. Conducting a deeper case study with several companies, studying their systems, is a possibility for the upcoming Master's Thesis. There will be meetings at the start of the next semester with the supervisor to discuss the work for the Master's Thesis.