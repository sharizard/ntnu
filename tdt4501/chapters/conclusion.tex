\chapter{Conclusion}
Technical debt is something that organizations are unable to avoid during software development projects. Technical debt is not always a bad thing to take. Organizations can use technical debt as a powerful tool to reach their customers faster and gain edge over the competition in the market. However, if technical debt is not paid back in time, it might generate economic consequences and quality issues to the software. It is necessary for organizations to create a strategy plan that includes practices and tools that decrese TD.

Empirical research has been performed to verify theories, or extend existing onces, and improve practice. This study is mainly used to gain understanding about technical debt accumulation in embedded systems with the goal to find the best practices for managing technical debt. This study reveals that the way traditional software developers and embedded system developer handle technical debt is different. The reason is that embedded system developers needs to deal with multiple constraints. We also found that the awareness of developers and managers is an important factor. When they are aware of the technical debt, it is less dangerous as it can be accounted for when planning for the future. Unknown and hidden debt is more dangerous. Having the ability to predict the long-term business outcomes of short-term technical decisions would help the software practitioners to choose the right kinds of technical debt to incur.

The results of the interview are presented. The literature study was used to provide a conceptual framework and understanding about the state-of-the-art, which defined the research questions.

\section{Future work}
A platform for managing technical debt? Make an app for crawling through your source code, visualize all modules and their dependices etc? Analysing tools like Git, Jira, find their weaknesses? Perform similar but bigger case studies in the Norwegian industry, or maybe another country?