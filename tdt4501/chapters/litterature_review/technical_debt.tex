\section{Technical debt}
Techincal debt is all about unfinished, unecessary complex, and outdated solutions which hinders us when it comes to maintencance of the solution in an effective way. 

Skjer når snarveies tar for å løse et IT-problem i stor grad fra kortsiktige hensyn til kostnad og tidsbruk. Medfører problemer ifm. Gjennomføring av endringer og nyutvikling hvis markedskravene endrer seg. Et økende globalt problem. 

Kan skapes av flere faktorer, deriblant teknologi, arbeidsprosesser (ifm utvikling), samhandling i organisasjon og mennesker (kunnskap og kapasitet).

Utviklere kan ikke bry seg pga mangel på eierskap til det som lages. Blir fortalt hva de skal gjøre, ikke med å utforme krav.

Vi ønsker å levere på tid. Spor opp den tekniske gjelden og nedbetal den før den skaper problemer.

Et stort problem idag. I 2010 lå den tekniske gjelden på 500 milliarder dollar. I 2015 vil den ligge på 1 billion dollar.

Kort tid til leveranse, alt for mye å gjøre. Levere en mindre god løsning i tide (quick and dirty), eller gjennomarbeidet løsning hvor vi går over tidsfristen. Quick and dirty gjør at vi leverer i tide, men på bekosnitng av design og kvalitet, som fører til at drift, forvaltning og vedlikeholdt blir komplekst. Å betjene teknisk gjeld kan være å vedlikeholde eksistrende kode enn å utvikle ny funksjonalitet. Det er lurt å sette opp en nedbetalingsplan som sier noe om hvordan gjelden skal nedbetales. Der kan man f.eks bruke scrum som utviklingsmetodikk hvor planen deles inn i estimerte og prioriterte oppgaver. Ikke ta opp for mye lån heller. Teknisk gjeld kan ses på som banklån ifølge Ward Cunningham. Man kan ta opp et huslån, men denne må nedbetales med renter. Å betjene TG tar masse tid. Akseptabelt å ta opp lån, men det bør være kontrollert. Ikke ta for mye lån enn det man klarer. Tenk med hodet. 

Teknisk gjeld kan forbindes med dårlig kode, men er som regel mye mer enn det. Som et vanlig lån kommer det renter med teknisk gjeld å, noe som ses på som farlig. Kostnadene man kan forbinde å ha med en gjeld. Renter kan være lavere utviklingstakt, lav konkurranseevne, dårlig sikkerhet mtp angrep utenfra, tap av utviklere og deres kompetanse, dårlig internt samarbeidsmiljø, misfornøyde kunder, tap av markedsandel. 
Skapes av fire faktorer:
-	Arbeidsprosesser
o	Utviklingsmetodikk, kan manuelle oppgaver automatiseres (deploy script f.eks), blir det skrevet tester etter feilretting, kartlegger og dokumenterer man snarveier man tar for å rekke deadline, legger man planer for håndtering av teknisk gjeld senere? Er det viktigere med ny funksjonalitet, eller håndtering av sånne ting? 
-	Mennesker (kunnskap og kapasitet)
o	Er man avhengig av enkeltpersoner? Har man de riktige personene som trengs til jobb? Blir det gitt god nok opplæring? Hva skjer når man er avhengig av noen som tar ferie, bytter prosjekt, blir sykemeldt, ut i permisjon eller pensjon? Tenk litt på det, og lag en plan på hvordan kompentasne skal bli overført. TG fører til lav motivasjon og produktivtet som igjen fører til lav effektivtet ved arbeid.
-	Teknologi
o	Er løsninger vanskelig å integrere med andre, er alle systemer kompatible med nye teknologier, finnes det utdaterte eller dupliserte koder i systemet, er systemene sikre mot angrep, løsninger gammeldags og brukervennlige, finnes det deler av systemet som er skrevet med gammel kode og vanskelig å vedlikeholde.
-	Samhandling i organisasjon
o	Kommunikasjon mellom kravstillere og utviklere, hvordan får man liste av funksjoner, er dette noe man lett skjønner, jobber vi med en backlog med oppgaver som skulle vært løst for lenge siden, men ikke er akutelle nå? 


Bruk av sekvensielle design prosesser som vannefallsmodellen for å lage store programvarer er ofte en failure. Krav spesifiseres i begynnelsen, og resten av perioden går ut på å følge disse kravene, som ikke kan endres. Lønner seg ikke når teknologi og business krav endrer seg stadig. Derfor ble agilt utviklingsmetodikk bearbeidet, handler om change and feedback. Problem med agilt: stor fokus på funksjonalitet, lite fokus på design, god kode, testing. Det fører til TG. 

Flere klassifiseringer på TG, som basert på livssyklus til prosjektfasen. Dokumentasjonsgjeld, designgjeld, kode, testing osv. 

Architecture violations might hinder future feature development as it might be hard to extend. 

------------------------------

In order to investigate the consequences of technical debt, we need to understand more formally what technical debt is, and how it occurs. When you're about to add new functionality to a system, you see two ways to do it. One is quick and dirty - you'll deliver the functionality in time but we trade off design and quality, which makes further changes harder in the future. The other way is a cleaner result, but in exhange of more time to put the functionality in place. The technical debt metaphor was first introduced by Ward Cunningham\cite{cunningham}. Like financial debt, the technical debt incurs interest payments, which come in the form of the extra effort that has to be done in future development because of bad design choices\cite{fowler}. This includes all aspects of software development lifecycle. 

\subsection{Types of technical debt}
There are many definitions of the types of technical debts that exists.


McConnels defines two categories based on how they are incurred, intentionally or unintentionally. The unintentional category includes debt that comes from doing a poor job. For instance, uninntentional debt might be when a junior software developer writes bad code due to lack of knowledge and experience. Intentional debt occurs when an organization makes a decision to optimize for the present rather than the future. An example is when the project release must be done on time, or else there wont be a next release. This leads to bad decisions, like taking a shortcut to solve a problem, and reconcile the problem after shipment

Fowlers presents a formal explanation of how techincal debt can occur. He categories technical debt into different debt types, in which he calls "Technical Debt Quadrant". As seen in the figure, the debt is grouped into four categories

Krutchen divides technical debt into two categories. Visible, debt that is visible for everyone. It containts elements such as new functionality to add and defects to fix. Invisible is the other category, debt that is only visible to software developers.



\subsection{Technical debt in Industry}

Klinger

Codabux
