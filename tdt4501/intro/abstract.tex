% !TEX encoding = UTF-8 Unicode
% !TEX root = ..\main.tex
% !TEX spellcheck = en-US
%%=========================================
\addcontentsline{toc}{section}{Abstract}

%==========
% Abstract
%==========
%1. State the problem.
%2. Say why it's an interesting problem.
%3. Say what your solution achieves.
%4. Say what follows from your solution.

% Context, objective, method, results, conclusion
\section*{Abstract}
\textbf{Context}: Today, software is contributing a substantial part of new functionalities and innovations of the embedded industry. With size and complexity of the software growing as time the time goes, additional challenges arises, including implicit assumptions of technical debt. Technical debt refers to situations where shortcuts are taken in technical decisions. Technical debt has been identified as one of the key reasons to software system projects failures. Accumulation of technical debt may reduce the dependability and maintainability of the embedded systems.

\textbf{Objective}: The goal of this study is to compare existing research on technical debt, and embedded systems, with the insights and experiences of traditional software and embedded system practitioners from the industry. We will explore and understand the causes of technical debt, as well as how it is managed in embedded systems.  

\textbf{Research Method}: Four different companies with one participant from each, were interviewed. Two of them works with traditional software development, while the last two works with embedded system development. All of them had different positions.

\textbf{Results}: The results consists of the knowledge similar to the technical debt research. There are many similarities between the causes and management practices of technical debt. Results show that technical debt is mostly formed as a result of intentional decisions to reach deadlines. However, embedded system has more control over technical debt than traditional software. The results also revealed that neither of the participants had any specific management plan for reducing technical debt but several practices were identified. The results may also help companies to understand the existing practices of technical debt management and use it to improve their own processes.