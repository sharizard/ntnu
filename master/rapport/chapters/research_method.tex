% !TEX encoding = UTF-8 Unicode
% !TEX root = ../main.tex
% !TEX spellcheck = en-US

\chapter{Research Methodology}

This chapter describes the research methodology that has been used in this thesis to answer the research questions. 

\section{Empirical Strategies}
According to Wohlin\cite{Wohlin:2000:ESE:330775}, empirical studies follows two types of research paradigms; the qualitative, and the quantitative paradigm. Qualitative research is concerned with studying objects in their natural setting\cite{Wohlin:2000:ESE:330775}. It is based on non-numeric data found in sources as interview tapes, documents, or developers' model\cite{Wohlin:2000:ESE:330775}. Quantitative research is concerned with quantifying a relationship or to compare two or more groups\cite{Wohlin:2000:ESE:330775}. It is based on collecting numerical data\cite{Wohlin:2000:ESE:330775}. 

Oates\cite{Oates:2006:RIS:1202299} presents six different research strategies; \textbf{survey}, \textbf{design and creation}, \textbf{case study}, \textbf{experimentation}, \textbf{action research}, and \textbf{ethnography}. \textbf{\textit{Survey}} focuses on collection data from a sample of individuals through their responses to questions. The primary means of gathering qualitative or quantitative data are interviews or questionnaires. The results are then analyzed using patterns to derive descriptive, explorative and explanatory conclusions. \textbf{\textit{Design and creation}} focuses on developing new IT products, or artifacts. It can be a computer-based system, new model, or a new method. \textbf{\textit{Case study}} focuses on monitoring one single 'thing'; an organization, a project, an information system, or a software developer. The goal is to obtain rich, and detailed data. \textbf{\textit{Experimentation}} are normally done in laboratory environment, which provides a high level of control. The goal is to investigate cause and effect relationships, testing hypotheses, and to prove or disprove the link between a factor and an observed outcome. \textbf{\textit{Action research}} focuses on solving a real-world problem while reflecting on the learning outcomes. \textbf{\textit{Ethnography}} is used to understand culture and ways of seeing of a particular group of people. The researcher spends time in the field by participating rather than observing.


\section{Research Design}

This study consists of one study

- Open source systems
- Object oriented

Data generation: Documents

Type: Descriptive, exploratory

Steps:
- Pre study (state of the art)
- Case study design
- Preparation for data collecton
- Collecting data
- Analysis of data
- Discussing the data            


Strategy:
- Literature review, and experience and motivation -> research questions -> case study -> documents -> qualitative/quantitative



\section{Case Study}


