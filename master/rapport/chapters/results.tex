% !TEX encoding = UTF-8 Unicode
% !TEX root = ../main.tex
% !TEX spellcheck = en-US

\chapter{Results}

This chapter presents the results of this study. 
In this study, we evaluated the design and software quality of the system. This study also addresses all of the research questions. 


%No documentation for pattern usage, but they have a document with coding standard. This includes how to use some patterns. But no docs on where patterns are used, not so big focus on it. 

%Hvor mye er gjenbrukt
%Finne klasser som brukes mye
%Se etter super-klasser
%Memory footprint
%Bruken av C++, gjenbruk kan skape arch drift
%Se etter kode som ikke blir brukt
%Kunne teste ting isolert

\section{Code Smell Detection}
\label{sub:code_smell_detection}
As we explained in Chapter 2, one of the ways to identify design debt is to look at the number of code smells in the source code. Table \ref{tab:identifiedCodeSmell} describes the number of code smells that were identified using automatic.

\begin{table}[]
\centering
\caption{Number of Code Smells detected}
\label{tab:identifiedCodeSmell}
\begin{tabular}{|l|l|}
\hline
\textbf{Code Smell}                           & \textbf{Detected}    \\ \hline
Long Method                                   & 10          \\ \hline
Large Class                                   & 8          \\ \hline
Long Parameter List                           & 15          \\ \hline
Data Clumps                                   & Bloaters          \\ \hline
Switch Statements                             & O-O Abusers       \\ \hline
Temporary Field                               & O-O Abusers       \\ \hline
Refused Bequest                               & O-O Abusers       \\ \hline
Alternative Classes with Different Interfaces & O-O Abusers       \\ \hline
Parallel Inheritance Hierarchies              & O-O Abusers       \\ \hline
Divergent Change                              & Change Preventers \\ \hline
Shotgun Surgery                               & Change Preventers \\ \hline
Lazy Class                                    & Dispensables      \\ \hline
Data Class                                    & Dispensables      \\ \hline
Duplicated Code                               & Approximately 5\% of the source code. 39 files affected.       \\ \hline
Speculative Generality                        & Dispensables      \\ \hline
\end{tabular}
\end{table}

\subsubsection{Duplicated Code}
Duplicated code is found by looking for a block of code that appears at multiple places in the source code, both internally in a file or in another file. Our results revealed that 39 files contains duplicated code. This determines 5\% of the source code, rougly 4400 lines of code is duplicated. Table XX summarizes the

\subsubsection{Large Class}

\subsubsection{Long Method}
Looks for LOC of the method. A long method code smell is considered to be 200 LOC. Another tool using the rules: 30 LOC and 150+ lines reported 47 hits. 



\subsubsection{Long Parameter List}
The maximum number of parameters allowed is set to 5, which means that 6 or more parameters in a method are considered as code smell. The various tools reported 15 hits, whereas 3 hits were considered as critical. The first critical hit contained nine parameters, and the other two contained twelve parameters. We verified the results manually by examining the class diagrams for the corresponding methods.

\subsubsection{Speculative Generality}

\subsubsection{Shotgun Surgery}


\subsubsection{Dead Code}
Dead Code is a code smell that is not mentioned by Fowler. TODO: Present dead code in Firmus.

Unused variables, unused functions, unnecessary header files. 

\subsubsection{God Class}
A god class is an object that controls way too many other objects in the system and has grown beyond all logic to become the class that does everything. One way to identify god classes is to look for the number of instance variables. 






How did we study the different code smells, the apporach and the results.
The results from Table XX

As we see, there are many code smells detected. We take a closer look at some of the classes; presented in UML diagrams here:





















\section{Traditional Software Quality Metrics Results}
\label{sub:QA_metrics_results}



\begin{table}[]
\centering
\caption{Project Summary}
\label{tab:projectsummary}
\begin{tabular}{|l|l|l|l|l|}
\hline
                                                    & \textbf{CCCC} & \textbf{SonarQube} & \textbf{Understand}                      & \textbf{SourceMonitor} \\ \hline
\textbf{Lines}                                      &               & 88404              & 88546                                    & 88546                  \\ \hline
\textbf{Lines of Code (LOC)}                        & 46220         & 48693              & 49287                                    &                        \\ \hline
\textbf{Number of Files (NOM)}                      &               & 461                & 461                                      & 461                    \\ \hline
\textbf{Number of Modules (NOF)}                    & 459           &                    &                                          &                        \\ \hline
\textbf{Classes}                                    &               & 851                & 830                                      & 375                    \\ \hline
\textbf{Functions}                                  &               & 3111               & 3589                                     & 812                    \\ \hline
\textbf{Statements}                                 &               & 45153              & 30109 (executable) + 18882 (declarative) & 35622                  \\ \hline
\textbf{McCabe's Cyclomatic Complexity (MVG)}       & 3572          & 15266              &                                          & Max complexity (71)    \\ \hline
\textbf{Comments (COM)}                             & 19705         & 15962              & 23017                                    & 22845                  \\ \hline
\textbf{LOC/COM (L\_C)}                             & 2.346         & 3.05               & 2.141                                    &                        \\ \hline
\textbf{MVG/COM (M\_C)}                             & 0.181         &                    &                                          &                        \\ \hline
\textbf{Information Flow Measure (inclusive) (IF4)} & 171493        &                    &                                          &                        \\ \hline
\textbf{Information Flow Measure (visible) (IF4v)}  & 166589        &                    &                                          &                        \\ \hline
\textbf{Information Flow Measure (concrete) (IF4c)} & 3789          &                    &                                          &                        \\ \hline
\end{tabular}
\end{table}


Table \ref{tab:projectsummary} summarizes the measurements over the project as a whole from the different tools. The results from the different tools reveals some differences in the measurements. For example, CCCC did not identify number of files in the project, but instead it identified number of non-trivial modules in the project. Non-trivial modules include all classes, and any other module for which member functions are identified. Furthermore, SonarQube identified 851 classes in the project. We can see that the difference between the results of amount of classes from SourceMonitor/Understand and SonarQube is big. In addition, SonarQube identified a total of 45153 statements in the source code, while Understand identified 35254 statements, which is pretty close to what SourceMonitor identified. 

Using CCCC, we were able to measure information flow between the different modules. This is done by identifying and counting inter-module couplings in the module interfaces. 

%MVG: A measure of the decision complexity of the function which make up the program. The strict definition if this measure is that it is the umber of linearly independent routes through a directed acyclic graph which maps the flow of control of a subprogram. The analyzer counts this by recording the number of distinct decision outcomes contained within each function, which yields a good approximation to the formally defined version of the measure.

%L_C : Lines per comment: Indicates density of comments with respect to logical complexity of program

%M_C: Indicates density of comments with respect to logical complexity of the program.

%IF4: Measure of information flow between modules suggested by Henry and Kafura. The analyzer makes an approx. count of this by counting inter-module couplings identified in the module interfaces.


Moreover, we gathered two types of metrics: Class metrics and file metrics. File metrics returns software quality metrics for the specific file, while class metrics contains metrics for each class. A file can contain multiple classes. We wish to look at the different metrics on each file to indicate if there is any weaknesses in that particular file and what classes the file contains. The metrics we measured are lines of code, complexity of file, average complexity per function, amount of functions, and max depth. 

SonarQube class metric includes nested classes, enums, interfaces, and annotations. Understand and SourceMonitor class metric includes classes and struct keyword. 

SonarQube has more accurate data on classes and functions, while SourceMonitor has more accurate data about statements. 






Metrics for files:
- Component / Filename
- Lines of Code
- Complexity
- Complexity per function
- Structure (classes, methods, statements)
- Functions
- Max depth



% Skal til Research Method


















\section{Object-Oriented Metrics in Firmus}
While traditional metrics are important for identifying large and complex classes, they alone may not tell us why some classes are big. In addition to the traditional metrics, we have also gathered data using object-oriented metrics. Object-oriented metrics we have used to measure the quality of the code is mostly based on the work of Chidamber and Kemerer.\cite{chidamber1994metrics}. They have proposed a set of static metrics that are designed to measure the quality of object-oriented software. These metrics are widely known, and their metrics suite is the deepest research in object-oriented metrics investigation and the measurements we have are the following: Weighted Method per Class (WMC), Depth of Inheritance Tree (DIT), Number of Children (NOC), Lack of Cohesion in Methods (LCOM), Response For a Class (RFC), and Coupling between Object Classes (CBO). 


\subsection{Object-Oriented Metrics for the whole Project}
A total of 321 files were analyzed. These files contains 229 classes, and 32068 lines of code. We have decided to exclude the tests from object-oriented metrics analysis. Descriptive statistics such as minimum, maximum, median, sample mean, and standard deviation are presented in this section. Table \ref{tab:oometrics-firmus} presents descriptive statistics for class level metrics for the whole project.

\begin{table}[]
\centering
\caption{OO-metrics for Project Firmus}
\label{tab:oometrics-firmus}
\begin{tabular}{|l|l|l|l|l|l|}
\hline
\textbf{Metric} & \textbf{Min} & \textbf{Max} & \textbf{Median} & \textbf{Sample Mean} & \textbf{Standard Deviation} \\ \hline
LCOM            & 0            & 100          & 55              & 42.205               & 33.042                      \\ \hline
DIT             & 0            & 4            & 1               & 1.061                & 1.062                       \\ \hline
CBO             & 0            & 30           & 5               & 6.079                & 5.179                       \\ \hline
NOC             & 0            & 20           & 0               & 0.454                & 1.850                       \\ \hline
RFC             & 0            & 115          & 10              & 15.777               & 18.677                      \\ \hline
WMC             & 0            & 48           & 7               & 8.616                & 7.167                       \\ \hline
NIM             & 0            & 48           & 7               & 8.376                & 6.983                       \\ \hline
NIV             & 0            & 18           & 1               & 2.223                & 2.811                       \\ \hline
\end{tabular}
\end{table}

TODO: Find X\% of the total classes with high coupling, low cohesion, deep hierarchy. 

A class is cohesive if LCOM is low. The median reveals that more than 50\% of the classes have low cohesion. These classes increases the complexity of the software, and may therefore increase the likelihood of errors during development. 

DIT value is generally low in the captured statistics. A class with DIT = 0 is the root of a class hierarchy. With an average value of 1, more than half of the classes inherits from a superclass. However, the statistics indicates some classes with deep hierarchy. Both average value of NOC and DIT are medium, showing that inheritance is used in most of the classes to an optimal level. This depth level is well managed at this point, and it probably comes from inheritance. Complexity is under control at this point in development. NOC median value tells us that approximately half of the classes have a flat class structure, indicating that inheritance is not used enough. However, the max NOC value is very large. Classes with high NOC value is difficult to modify and will require more testing because of the effects on changes on all the children. 

In general, higher values of CBO indicates fault prone classes. Both sample mean and median shows low CBO values for over half of the classes in this system. However, the maximum value that has been captured is very large. This class is an example of a class that is hard to understand, harder to reuse, and more difficult to maintain. 

The RFC statistics reveals that most classes have a RFC of less than 10. However, the maximum value is revealed to be 115. Classes with large RFC tends to be complex and have decreased understandability. Testing classes with large RFC is more complicated. In addition, most of the classes have a WMC of less than 7, but there are a few classes with more extreme values. Those classes with highest WMC are candidates for inspection and refactoring. 


%Good refactoring would be to look at all the classes and see where we could use more abstraction. Reducing redundancy would reduce code size and speed. 

\subsection{Object-Oriented Metrics for the Components}
Descriptive statistics in Table \ref{tab:oometrics-firmus} reveals statistics for class level metrics for the whole project. However, the statistics does not say anything about class level metrics in the different components. Some components may have good statistics, while other components have bad statistics In order to identify the weak components, we calculated descriptive statistics for each component, presented in the coming subsubsections. 

\subsubsection{Component A}
Excluding the tests, we counted 56 files containing 40 classes and 6286 lines of code. Descriptive statistics for Component A are presented in Table \ref{tab:oometrics-al}.
\begin{table}[]
\centering
\caption{OO-metrics for component A}
\label{tab:oometrics-al}
\begin{tabular}{|l|l|l|l|l|l|}
\hline
\textbf{Metric} & \textbf{Min} & \textbf{Max} & \textbf{Median} & \textbf{Sample Mean} & \textbf{Standard Deviation} \\ \hline
LCOM            & 0            & 94           & 57              & 42.925               & 35.222                      \\ \hline
DIT             & 0            & 4            & 1               & 1.525                & 1.132                       \\ \hline
CBO             & 0            & 29           & 5               & 5.875                & 6.252                       \\ \hline
NOC             & 0            & 8            & 0               & 0.7                  & 1.652                       \\ \hline
RFC             & 2            & 115          & 28.5            & 40.525               & 32.252                      \\ \hline
WMC             & 2            & 44           & 10.5            & 12.675               & 9.339                       \\ \hline
NIM             & 2            & 40           & 10              & 12.3                 & 8.979                       \\ \hline
NIV             & 0            & 12           & 1               & 2.4                  & 2.889                       \\ \hline
\end{tabular}
\end{table}

Among the 40 classes, over half of the classes has low cohesion. 
DIT and NOC is very low.
CBO has average values, but one of the classes has a maximum value of 8.
RFC median shows, but the average value is higher than the median. This indicates that some classes have high RFC values, hence increasing the sample mean. WMC is set to be low, but some classes has high values.
NIM and NIV is moderate.












\subsubsection{Component B}
Excluding the tests, we counted 42 files containing 23 classes and 3905 lines of code. Descriptive statistics for Component B are presented in Table \ref{tab:oometrics-al}.
\begin{table}[]
\centering
\caption{OO-metrics for Component B}
\label{tab:oometrics-blc}
\begin{tabular}{|l|l|l|l|l|l|}
\hline
\textbf{Metric} & \textbf{Min} & \textbf{Max} & \textbf{Median} & \textbf{Sample Mean} & \textbf{Standard Deviation} \\ \hline
LCOM            & 0            & 100          & 55              & 42.205               & 33.042                      \\ \hline
DIT             & 0            & 4            & 1               & 1.061                & 1.061                       \\ \hline
CBO             & 0            & 30           & 5               & 6.079                & 5.179                       \\ \hline
NOC             & 0            & 20           & 0               & 0.454                & 1.850                       \\ \hline
RFC             & 0            & 115          & 10              & 15.777               & 18.677                      \\ \hline
WMC             & 0            & 48           & 7               & 8.616                & 7.156                       \\ \hline
NIM             & 0            & 48           & 7               & 8.375                & 6.983                       \\ \hline
NIV             & 0            & 86           & 1               & 2.222                & 2.811                       \\ \hline
\end{tabular}
\end{table}

LCOM is very high. Similar to Component A, half of the classes have low cohesion. 

DIT and NOC has a very low sample mean and median. However, max NOC value is 20.

RFC has average values, but max is very high. 

CBO is average, but max value is set to 30.

WMC is average, but max is high. 

NIM and NIV. 






\subsubsection{Component C}
Component C contains 30 files, 20 classes, and 4763 lines of code. 
\begin{table}[]
\centering
\caption{OO-metrics for Component C}
\label{tab:oometrics-config}
\begin{tabular}{|l|l|l|l|l|l|}
\hline
\textbf{Metric} & \textbf{Min} & \textbf{Max} & \textbf{Median} & \textbf{Sample Mean} & \textbf{Standard Deviation} \\ \hline
LCOM            & 0            & 99           & 61              & 55.7                 & 23.58                       \\ \hline
DIT             & 0            & 1            & 0               & 0.1                  & 0.308                       \\ \hline
CBO             & 1            & 18           & 4               & 5.55                 & 4.662                       \\ \hline
NOC             & 0            & 0            & 0               & 0                    & 0                           \\ \hline
RFC             & 3            & 26           & 8.5             & 10.3                 & 5.741                       \\ \hline
WMC             & 3            & 26           & 8.5             & 10.3                 & 5.741                       \\ \hline
NIM             & 3            & 26           & 8.5             & 9.85                 & 5.153                       \\ \hline
NIV             & 0            & 9            & 2               & 3.15                 & 3.013                       \\ \hline
\end{tabular}
\end{table}

LCOM median shows that half of the classes has more than 60\% LCOM. 

DIT and NOC is very low. Inheritance is not used that much. 

CBO has moderate values, but maximum value is relatively high. 

RFC moderate values.

WMC: Max is a bit far away from sample mean and average.

NIM and NIV






\subsubsection{Component D}
Component D has 13 files with 1647 lines of code. Among these, we found one class. 
\begin{table}[]
\centering
\caption{OO-metrics for Component D}
\label{tab:oometrics-dist}
\begin{tabular}{|l|l|l|l|l|l|}
\hline
\textbf{Metric} & \textbf{Min} & \textbf{Max} & \textbf{Median} & \textbf{Sample Mean} & \textbf{Standard Deviation} \\ \hline
LCOM            & 68           & 68           & -               & 68                   & -                           \\ \hline
DIT             & 1            & 1            & -               & 1                    & -                           \\ \hline
CBO             & 7            & 7            & -               & 7                    & -                           \\ \hline
NOC             & 0            & 0            & -               & 0                    & -                           \\ \hline
RFC             & 8            & 8            & -               & 8                    & -                           \\ \hline
WMC             & 8            & 8            & -               & 8                    & -                           \\ \hline
NIM             & 8            & 8            & -               & 8                    & -                           \\ \hline
NIV             & 2            & 2            & -               & 2                    & -                           \\ \hline
\end{tabular}
\end{table}






\subsubsection{Component En}
3 files, 367 lines of code. Only 1 class.
\begin{table}[]
\centering
\caption{OO-metrics for Component En}
\label{tab:oometrics-en}
\begin{tabular}{|l|l|l|l|l|l|}
\hline
\textbf{Metric} & \textbf{Min} & \textbf{Max} & \textbf{Median} & \textbf{Sample Mean} & \textbf{Standard Deviation} \\ \hline
LCOM            & 62           & 62           & -               & 62                   & -                           \\ \hline
DIT             & 0            & 0            & -               & 0                    & -                           \\ \hline
CBO             & 1            & 1            & -               & 1                    & -                           \\ \hline
NOC             & 0            & 0            & -               & 0                    & -                           \\ \hline
RFC             & 8            & 8            & -               & 8                    & -                           \\ \hline
WMC             & 8            & 8            & -               & 8                    & -                           \\ \hline
NIM             & 8            & 8            & -               & 8                    & -                           \\ \hline
NIV             & 2            & 2            & -               & 2                    & -                           \\ \hline
\end{tabular}
\end{table}




\subsubsection{Component Ex}
48 files, 4089 lines of code. 86 classes.
\begin{table}[]
\centering
\caption{OO-metrics for Component Ex}
\label{tab:oometrics-ex}
\begin{tabular}{|l|l|l|l|l|l|}
\hline
\textbf{Metric} & \textbf{Min} & \textbf{Max} & \textbf{Median} & \textbf{Sample Mean} & \textbf{Standard Deviation} \\ \hline
LCOM            & 0           & 100          & 0               & 25.988               & 32.905                      \\ \hline
DIT             & 0            & 3            & 2               & 1.581                & 1.121                       \\ \hline
CBO             & 0            & 16           & 4               & 4.919                & 4.018                       \\ \hline
NOC             & 0            & 20           & 0               & 0.744                & 2.736                       \\ \hline
RFC             & 0            & 28           & 8               & 10.279               & 6.030                       \\ \hline
WMC             & 0            & 22           & 3.5             & 5.07                 & 3.928                       \\ \hline
NIM             & 0            & 22           & 3               & 4.907                & 3.846                       \\ \hline
NIV             & 0            & 10           & 0               & 1.209                & 2.098                       \\ \hline
\end{tabular}
\end{table}







\subsubsection{Component G}
59 files, 3701 lines of code, 32 classes.
\begin{table}[]
\centering
\caption{Component G}
\label{tab:oometrics-guri}
\begin{tabular}{|l|l|l|l|l|l|}
\hline
\textbf{Metric} & \textbf{Min} & \textbf{Max} & \textbf{Median} & \textbf{Sample Mean} & \textbf{Standard Deviation} \\ \hline
LCOM            & 0            & 94           & 60              & 50.25                & 31.236                      \\ \hline
DIT             & 0            & 2            & 1               & 0.625                & 0.609                       \\ \hline
CBO             & 0            & 22           & 5.5             & 6.187                & 4.987                       \\ \hline
NOC             & 0            & 2            & 0               & 0.25                 & 0.622                       \\ \hline
RFC             & 2            & 30           & 9               & 10.187               & 6.382                       \\ \hline
WMC             & 2            & 30           & 7.5             & 8.594                & 5.405                       \\ \hline
NIM             & 0            & 29           & 7               & 8.437                & 5.459                       \\ \hline
NIV             & 0            & 18           & 2               & 3.062                & 3.926                       \\ \hline
\end{tabular}
\end{table}










\subsubsection{Component L}
16 files, 849 lines of code, 7 classes.
\begin{table}[]
\centering
\caption{OO-metrics for Component L}
\label{tab:oometrics-log}
\begin{tabular}{|l|l|l|l|l|l|}
\hline
\textbf{Metric} & \textbf{Min} & \textbf{Max} & \textbf{Median} & \textbf{Sample Mean} & \textbf{Standard Deviation} \\ \hline
LCOM            & 0            & 80           & 58              & 50.857               & 35.130                      \\ \hline
DIT             & 0            & 1            & 1               & 0.571                & 0.534                       \\ \hline
CBO             & 1            & 13           & 4               & 5.571                & 4.197                       \\ \hline
NOC             & 0            & 0            & 0               & 0                    & 0                           \\ \hline
RFC             & 5            & 12           & 9               & 8.571                & 2.936                       \\ \hline
WMC             & 5            & 12           & 9               & 8.571                & 2.936                       \\ \hline
NIM             & 3            & 12           & 7               & 8.286                & 3.402                       \\ \hline
NIV             & 0            & 5            & 1               & 1.571                & 2.070                       \\ \hline
\end{tabular}
\end{table}








\subsubsection{Component N}
17 files, 1839 lines of code, 8 classes.
\begin{table}[]
\centering
\caption{OO-metrics for Component N}
\label{tab:oometrics-netw}
\begin{tabular}{|l|l|l|l|l|l|}
\hline
\textbf{Metric} & \textbf{Min} & \textbf{Max} & \textbf{Median} & \textbf{Sample Mean} & \textbf{Standard Deviation} \\ \hline
LCOM            & 0            & 79           & 70.5            & 54.5                 & 33.899                      \\ \hline
DIT             & 0            & 1            & 0               & 0.25                 & 0.463                       \\ \hline
CBO             & 3            & 17           & 10.5            & 10                   & 4.140                       \\ \hline
NOC             & 0            & 1            & 0               & 0.125                & 0.353                       \\ \hline
RFC             & 6            & 32           & 9               & 11.625               & 8.568                       \\ \hline
WMC             & 6            & 23           & 9               & 10.5                 & 5.580                       \\ \hline
NIM             & 6            & 21           & 8.5             & 9.75                 & 5.036                       \\ \hline
NIV             & 0            & 8            & 5.5             & 4.375                & 3.068                       \\ \hline
\end{tabular}
\end{table}




\subsubsection{Component P}
12 files, 722 lines of code, 8 classes.
\begin{table}[]
\centering
\caption{OO-metrics for Component P}
\label{tab:oometrics-proc}
\begin{tabular}{|l|l|l|l|l|l|}
\hline
\textbf{Metric} & \textbf{Min} & \textbf{Max} & \textbf{Median} & \textbf{Sample Mean} & \textbf{Standard Deviation} \\ \hline
LCOM            & 0            & 81           & 65              & 58.25                & 26.611                      \\ \hline
DIT             & 0            & 1            & 0               & 0.25                 & 0.463                       \\ \hline
CBO             & 0            & 12           & 5.5             & 5.5                  & 3.964                       \\ \hline
NOC             & 0            & 1            & 0               & 0.125                & 0.353                       \\ \hline
RFC             & 2            & 14           & 6.5             & 7.5                  & 3.625                       \\ \hline
WMC             & 2            & 14           & 6.5             & 7.25                 & 3.412                       \\ \hline
NIM             & 2            & 13           & 6.5             & 7                    & 3.117                       \\ \hline
NIV             & 0            & 6            & 3.5             & 3.125                & 2.031                       \\ \hline
\end{tabular}
\end{table}





\subsubsection{Component S}
4 files, 223 lines of code, 2 classes.
\begin{table}[]
\centering
\caption{OO-metrics for Component S}
\label{tab:oometrics-sys}
\begin{tabular}{|l|l|l|l|l|l|}
\hline
\textbf{Metric} & \textbf{Min} & \textbf{Max} & \textbf{Median} & \textbf{Sample Mean} & \textbf{Standard Deviation} \\ \hline
LCOM            & 33           & 50           & 41.5            & 41.5                 & 12.021                      \\ \hline
DIT             & 0            & 1            & 0.5             & 0.5                  & 0.707                       \\ \hline
CBO             & 3            & 6            & 4.5             & 4.5                  & 2.121                       \\ \hline
NOC             & 0            & 0            & 0               & 0                    & 0                           \\ \hline
RFC             & 6            & 9            & 7.5             & 7.5                  & 2.121                       \\ \hline
WMC             & 6            & 9            & 7.5             & 7.5                  & 2.121                       \\ \hline
NIM             & 6            & 9            & 7.5             & 7.5                  & 2.121                       \\ \hline
NIV             & 1            & 1            & 1               & 1                    & 0                           \\ \hline
\end{tabular}
\end{table}





\subsubsection{Component W}
1 file, 1 class, 69 lines of code.
\begin{table}[]
\centering
\caption{OO-metrics for Component W}
\label{tab:oometrics-watch}
\begin{tabular}{|l|l|l|l|l|l|}
\hline
\textbf{Metric} & \textbf{Min} & \textbf{Max} & \textbf{Median} & \textbf{Sample Mean} & \textbf{Standard Deviation} \\ \hline
LCOM            & 58           & 58           & -               & 58                   & -                           \\ \hline
DIT             & 0            & 0            & -               & 0                    & -                           \\ \hline
CBO             & 4            & 4            & -               & 4                    & -                           \\ \hline
NOC             & 0            & 0            & -               & 0                    & -                           \\ \hline
RFC             & 6            & 6            & -               & 6                    & -                           \\ \hline
WMC             & 6            & 6            & -               & 6                    & -                           \\ \hline
NIM             & 6            & 6            & -               & 6                    & -                           \\ \hline
NIV             & 2            & 2            & -               & 2                    & -                           \\ \hline
\end{tabular}
\end{table}







Table XX lists the classes and their metrics we thought were interesting. 


- LOC (Lines of Code)
- WMC (Weighted Methods Per Class)
- DIT (Depth of Inheritance Tree)
- NOC (Number of Children)
- Coupling (Various coupling metrics, such as afferent and efferent)
- LCOM (lack of cohesion)
- CCN (McCabe Cyclomatic Complexity)
- CBO (Coupling between Objects)

Table X shows a sample of metrics for some of the classes. 


A description of Software Metrics can be found in Section 2 in Chapter 2. 

Table X shows the value of each software metric captured. For some of the metrics, the lower value, the better. Values that are presented in the table are average. The metrics calclation are from several static code analyzers, both open source and commercial: CppDepend, SourceMonitor, CCCC, Understand. All these software provides extensive numbers of software quality metrics. We will show the metrics that are related to the code smells; hence we do not show all the metrics here. 

Our results show that (whats good and wrong with these metrics, what can be done). 


% HVA SLAGS METRIKKER ER INTERESSANT, GÅ DYPERE INN PÅ DEM. F.EKS METHODS IN CLASS, HVA SIER DET OSS

% TOO MANY LINES OF CODE? CHECK SINGLE RESPONSBILITY PRINCIPLE, it states that every class or module should have responsbility for a single part of the funcitnaility provided by the software.'































