% !TEX encoding = UTF-8 Unicode
% !TEX root = ../main.tex
% !TEX spellcheck = en-US

\chapter{Introduction}

This chapter provides an introduction to this masters thesis. We begin with outlining the motivation and context for the research. Then a brief description of the research questions is presented. Thesis outline is presented in the last section.


\section{Motivation}
Successful embedded systems continuously evolve in response to external demands for new functionality and big fixes\cite{graaf2003embedded}. One consequence of such evolution is an increase of issues in design, development, and maintainability\cite{7381510}. Software code often ends up not contributing to the mission of the original intended software architecture or design. The main challenge with software evolution is the technical debt that is not paid by the organization during software development and maintenance. Technical debt addresses the debt that software developers accumulate by taking shortcuts in development in order to meet the organizations business goals. For example, a deadline may lead developers to create "non-optimal" solutions in order to deliver on time. When technical debt keeps accumulating, systems can become unmanageable and eventually unusable. More resources during software maintenance have to be spent on paying off the interest (the cost of having the debt). According to Gartner\cite{gartner2010}, the cost of dealing with technical debt threatens to grow to \$1 trillion globally by 2015. That is the double of the amount of technical debt in 2010. Furthermore, many embedded systems are getting interconnected within existing Internet infrastructure. This is known as Internet of Things. This has led to embedded systems threatened by security issues. Matthew Garret recently revealed that he had access to the electronic equipment connected to a network in every hotel room in a hotel located in London.

Several studies has classified the metaphor of technical debt into different types of debt that are associated with the different phases of software development("siter noen artikler her"). Design debt is an example of a type of technical debt, which accumulates when compromises are made in software architecture level. Software design plays a significant role in the development of large systems\cite{krutchen}, and unlike code-level debt, design debt usually has more significant consequences\cite{mo2013mapping}. 


%Embedded software complexity continues to grow steadily. Many embedded systems were developed many years ago, and even with software evolution, the architecture and design has not changed. One consequence with embedded software evolution is increases of issues in design, development, and maintainability. 

%With the increasing demand for features for embedded systems, complexity of embedded software has been on the rise





%In this context, the quality of embedded software becomes crucial. 


\section{Research Context}
This master thesis builds upon our previous study "Managing Technical Debt in Embedded Systems"\cite{forprosjekt}, a prestudy that was carried out in the fall of 2015. The written assignment for the specialization project had the following definition:

\begin{addmargin}[2em]{2em}
\textbf{Managing Technical debt in embedded systems} \\
"This task is related to management of software in embedded systems, as well as evolution of such software over time. Embedded systems have often a long lifetime and it is thus important to find out best practices and tools for this management since it is necessary to cope with architectural and design decisions which were made perharps decades ago, as well as clearly find out how present decisions may affect future maintenance and operation. This is called technical debt since all decisions will have a future cost related to them. Such decisions are often not documented, the people that made the software is not available 10-20 years after the implementation, the Internet of Things make all kind of embedded systems accessible from the Internet and thus posing security threats.

The project may take different directions based on the students interests and motivation. Industrial companies are very interested in this topic, so it is possible to study industrial systems, both past and current., make suggestions and implement them, make tools, make processes, make best practice etc."
\end{addmargin}

In our previous research, we did a prestudy of the field of technical debt in embedded systems, where we investigated the reasons for companies to incur technical debt and the different strategies for managing it. Data was collected by conducting semi-structured interviews. After completing the study, we had a desire to look into a more narrow field of the concept technical debt by conducting a deeper case study for our upcoming master thesis. An interest was to study an industrial system.

The work in our master thesis has been done in collaboration with Autronica Fire and Security AS, a global provider of safety solutions which includes fire safety equipment, marine safety monitoring, and surveillance equipment. Their main office are located in Trondheim, one of the largest cities in Norway. We have performed a deeper study on one of the company's fire detection systems software for approximately six weeks. The study explored their source code in order to identify design flaws. 


\section{Research Design and Questions}
\label{sec:chap1designquesitons}
The goal of the analysis is to identify design flaws in their software before it gets worse. The relevant research methods in software engineering can be survey, design and creation, case study, experimentation, action research, and ethnography\cite{Oates:2006:RIS:1202299}. In this study, literature review and case study have been used to answer our research questions. Literature review was a part of the pre-study and has been used to get familiar with the term design debt and to define the research questions. Case studies are empirical methods used to investigate a single entity or phenomenon within a specific time space\cite{Wohlin:2000:ESE:330775}, which fits our desire to study an industrial system. Case studies can be both qualitative and quantitative\cite{bassey2003case,Oates:2006:RIS:1202299}. The research process we have chosen to adopt in this study follows the principles of the six steps defined by Soy\cite{soysusan}: \textit{Determine and Define The Research Questions, Select the Cases and Determine Data Gathering and Analysis Techniques, Prepare to Collect Data, Data Collection, Evaluate and Analyze Data, and Prepare the Report}.

The main research questions investigated in this thesis are:

\begin{itemize}
	\item \textbf{RQ1}: How can design debt be identified?
	%\item \textbf{RQ2}: How can architectural technical debt be re-factored to improve the quality?
	\item \textbf{RQ2}: What kind of design debt can be found in embedded systems?
	\item \textbf{RQ3}: What are the effects of design debt?
	\item \textbf{RQ4}: How to pay design debt?
\end{itemize}

\section{Contrubution}
TODO: Contribution: New knowledge about design debt in embedded software. How it differs from existing research. Will be written at the end of this study.


\section{Thesis Structure}
The thesis is structured into several chapters with sections and subsections. The outline of the thesis is as follows:
\begin{itemize}
	\item{\textbf{Chapter 1}}: Introduction contains a brief and general introduction to the study and the motivation behind it.
	\item{\textbf{Chapter 2}}: State-of-the-Art looks at important aspects of the research question.
	\item{\textbf{Chapter 3}}: Research Method describes how the literature review was carried out throughout the research, as well as a description of the case study to be performed.
	\item{\textbf{Chapter 4}}: Results presents the results from the case study, and takes a closer look at the findings from the case study.
	\item{\textbf{Chapter 5}}: Discussion contains a summarized look at the findings from the case study, and connects it with the literature review and to the research questions. An evaluation of the research is also given in this chapter.
	\item{\textbf{Chapter 6}}: Conclusion concludes the research by providing a summary of the most important points of the results and discussion chapter. Additionally, in outlines possible routes to take in the research field.
\end{itemize}

