% !TEX encoding = UTF-8 Unicode
% !TEX root = ..\main.tex
% !TEX spellcheck = en-US

\chapter{Introduction}

This chapter provides an introduction to this masters thesis. We begin with outlining the motivation and context for the research. Then a brief description of the research questions is presented. Thesis outline is presented in the last section.


\section{Motivation and Background}
The field of embedded systems is growing rapidly based on the evolution in electronics and widespread use of sensors and actuators. From consumer electronics, automobiles, to satellites, embedded systems represent one of the largest segments of the software industry. Software plays a central role in the development of embedded systems. Embedded software is the primary driving force for implementing different functionalities of todays embedded systems. The software is specialized for one particular hardware, and may therefore introduce hardware specific run-time constraints. Embedded systems are growing exponentially\cite{graaf2003embedded}, and is raising issues in design, development, and maintainability. The main challenge is the technical debt that is not paid by the organization during the software life cycle. Technical debt addresses the debt that software developers accumulate by taking shortcuts in development to meet the organizations business goals. For example, early software release versus maintainability. According to Gartner\cite{gartner2010}, the cost of dealing with technical debt threatens to grow to \$1 trillion globally by 2015. That is the double of the amount of technical debt in 2010. 

%In this context, the quality of embedded software becomes crucial. 


\section{Research Context}


\section{Research Questions}


\section{Thesis Structure}
The thesis is structured into several chapters with sections and subsections. The outline of the thesis is as follows:
\begin{itemize}
	\item{\textbf{Chapter 1}}: Introduction contains a brief and general introduction to the study and the motivation behind it.
	\item{\textbf{Chapter 2}}: State-of-the-Art looks at important aspects of the research question.
	\item{\textbf{Chapter 3}}: Research Method describes how the literature review was carried out throughout the research, as well as a description of the case study to be performed.
	\item{\textbf{Chapter 4}}: Results presents the results from the case study, and takes a closer look at the findings from the case study.
	\item{\textbf{Chapter 5}}: Discussion contains a summarized look at the findings from the case study, and connects it with the literature review and to the research questions. An evaluation of the research is also given in this chapter.
	\item{\textbf{Chapter 6}}: Conclusion concludes the research by providing a summary of the most important points of the results and discussion chapter. Additionally, in outlines possible routes to take in the research field.
\end{itemize}

