% !TEX encoding = UTF-8 Unicode
% !TEX root = ..\main.tex
% !TEX spellcheck = en-US

\chapter{Conclusion}
This thesis presents the result of a case study that has investigated design debt in embedded systems. Design debt is a major problem for many software projects today. 


The purpose of this study was to identify design debt in embedded systems by measuring object-oriented metrics and detecting code smells. The results show that **. 

The main contributions of this work are:

Being able to measure and monitor the quality of software design will help the team to understand the current situation of design debt by making them visible. Once the problems are visible, suitable actions can be performed if needed. This includes refactoring and eventually re-engineering. 

\section{Future Work}
Based on the results from this thesis, we outline some possibilites for future research.

\begin{itemize}
	\item Measuring other object-oriented metrics
	\item Implementation of tools which can suggest refactoring options
	\item A deeper study on multiple release version of a system to compare the evolution of metrics and patterns over time.
	\item Study design pattern
	\item A study on differences between metrics thresholds measured in this study and data from other similar systems. For example, it may be possible to compare our restuls with the measurement of thresholdts of open-source systems. 
\end{itemize}
