\section*{Research plan}
My research is divided into two phases, a specialization project in the fall 2015 and master thesis in the following spring. This research plan will cover both fases.

\subsection*{Purpose}
%You should convince the reader about the purpose of your project by motivating and referring to existing literature that support your thesis/hypothesis. You should also clarify your research questions justified by the purpose. (35\%)

%"Describe the reason for doing the research, the topic of interest, why it is important or useful to study this, the specific research question(s) asked and the objectives set. Research without a purpose is unlikely to be good research." (Oates 2005)

%Write in this section about the following:
%- What is the problem? Whose problem is it? Refer to 2-3 good scientific references that confirm for the reader that this is actually a relevant problem.
%- What is done earlier to address this problem? Give 4-5 good references that illustrate the different approaches taken by other researchers to solve this problem. (you might %also consider to do an in-depth literature review as part of your autumn project).
%- What is wrong with earlier research? Did they not manage to solve the problem? Why? Why is your approach going to be better? What new knowledge do you plan to add?
%- Summarize with a list of clearly defined research questions. One main question and 2-3 sub-questions related to the main question. Note that the sub-questions should contribute to answer the main question.



\subsection*{Contributions}
%You should demonstrate that you have a clear understanding of what new knowledge or other new products, processes, models, theories etc. you add to the existing knowledge. (20%)

%"Describe the outcomes of research, especially your contribution to knowledge about your subject area. Your contribution can be an answer to your original research question( s) but can also include unexpected findings. For example, you and the academic community might learn something about a particular research strategy as a result of your research. Your thesis, dissertation, conference paper or journal article is also a product of your research. For those research projects that involve design and creation, a new computer-based product or new development method could also be a product of your research." (Oates 2005)

%Types of research products: 1) new or improved evidence; 2) new or improved methodology; 3) new or improved analysis; 4) new or improved concepts or theories; 5) new or improved computer-based product.

%Write in this section about the following:
%- What will be your research contribution? Which of the four types above
%- How and why will it be better than earlier contributions?

The main goal of this research is to develop a computer-based product that will support the embedded software industry to manage technical debt that is present in their products. By managing technical debt companies can enjoy increased profits by balacining investments on the quality of software and shorter-times to market or other forms of financial leverage. Although the main focus is on embedded software, the derived methodologies and tools will be applicable to software systems in general by configuring input parameters appropiately. 

Existing technical debt management approaches do not deal with the specifics of embedded systems development. 

De fleste strategier og practices for håndtering av teknisk gjeld idag er relatert mot kode. Noen analyserer kostnadene og måler tekniske gjelden. Det finnes et verktøy, men er mer rettet mot kode. Å ha det i backlogen er også en løsning. Men å ha et verktøy som kan si noe om type gjeld du har, hva kostnaden til den, hvor kritisk den er. Et platform som både sjekker kode, og all annet materiale som krav, design, arkitektur, bug reports osv.


\subsection*{Research Method}
%You need to show and convince your reader about the decisions you have made regarding research strategy and data generation methods. Strategy and data generation methods are related to your research questions as shown in Figure 3.1 in the book. You can of course have other strategies and data generation methods (the books gives only a number of examples) as long as you justify your decisions. (20\%)

%"Describe the sequence of activities undertaken in any research project. The process involves identifying one or more research topics, establishing a conceptual framework (the way you choose to think about your research topic), the selection and use of a research strategy and data generation methods, the analysis of data and the drawing of conclusions, including recognizing any limitations in your own research. As explained already, the process should be carried out systematically if the research is to be accepted as rigorous."

%Write here your (See the figure below from Oates 2005):
%- Research strategy (survey, case study, action research, experiment, design etc.)
%- Is this mixed-method? Then you can have different strategies for different steps in the design.
%- Data generation methods (interview, focus group, observation, documents etc.)
%- You can have more than one data generation method for each strategy.
%- Data analysis methods (qualitative or quantitative.)

To find the research questions, a literature review was conducted. 


\subsection*{Participants}
%If you plan to evaluate your thesis with users you need to describe this. Any other dependency on people or resources (e.g. experts you need to interview, data sets you need access to) needs to be written here. (10\%)

%"These include those whom you directly involve in your research, for example by interviewing them or observing them, and also those who are indirectly involved, such as the editors to whom you submit a research paper. It is important that you deal with all these people legally and ethically, that is, you do not do anything that might annoy them or cause them harm (physically, mentally or socially). You yourself as a researcher are also a research participant. As we shall see later, for some types of research, researchers are expected to be objective and remain largely unseen in the reporting of their research, whereas in other types of research the researchers are open about their feelings and how their presence influenced the other participants and the research situation."

%You need to clarify the following:
%- Who are the informants? How do you plan to recruit them?
%- If you want to collect data you should take care of the ethical issues. E.g. 1) you should get approval from NSD before you start collecting data, 2) you should have users sign informed consent forms, 3) you should make sure you don't share person-related data (e.g. you cannot have this data on Google doc!).
%- Don't underestimate use recruitment. It is often the part of the research that shows to be the most problematic! Get started early.
%- Who are the researchers?
%- Be specific who will do what in the project.


As a researcher, I am included as a participant. My work is to plan and conduct the research.

My main supervisor is Carl-Fredrik Sørensen. My research will be supervised by him. He is contributing with his experience in the software engineering field. 

In this research, the aim is to collect data from individuals. Every developer

\subsection*{Research Paradigm}
Interpretivism 
 %You should read the related chapters in the book and try to describe what paradigm you are in, positivist, interpretivism or critical research. Discuss this with your supervisor. (10\%)

 %"A pattern or model or shared way of thinking. Managers sometimes talk of the need for a ‘paradigm shift’ to mean that a new way of thinking is required. In computing, we talk about programming language paradigms, for example, a group of languages that share a set of characteristics, such as the object-oriented paradigm (for example, Smalltalk and C + +). Here we are concerned with the philosophical paradigms of research. Any piece of research will have an underlying paradigm. We have noted already that different academic communities and individuals have different ideas about the kinds of research questions to ask and the process by which to answer them because they have different views about the nature of the world we live in and therefore about how we might investigate it. These different views stem from different philosophical paradigms. We shall look at three such paradigms: ‘positivism’, interpretivism’ and ‘critical research’ – each will be explained later."

Survey har blitt valgt som strategi i denne forskningen. Ved å bruke intervju som datagenerasjonsmetode får vi kvalitattive data som resultat. Mens posivisme ønsker å få et generelt overblikk, noe som fører til høy validiter, utføres intervjuer i mer detalje og ser på kulture og hvordan folk lever livet. Forskeren kan derfor føle hvordan intervju kandidaten opplever et problem. Dataen som fås er detaljert og gir dypere forklaring på et problemDette for å utforsoske, forklare og forstå realiteten. Denne måten å tenke på tilsvarer interpretivisme. 

Surveys are usually storngly associated with positivsm, as it seeks patterns in the world. However, surveys can be used in a more interpretive way. A survey could be carried out among people who differ from each other. This might enable the researchers to establish the breadth of opninions about a subject. For example, they might try to find out about the many differnet factors thought to be relevant to successful UT systems implementation, without attempting to draw any conclusiion about the most popular factors.

Subjektive data tilbake fra respondenter. 



Need to explore, explain and undestand reality.



\subsection*{Final Deliverables and Dissemination}
%Will this be a report, a video, a demonstration, presentation etc.? How do you justify the presentation form you have chosen? (5\%)

%"The means by which the research is disseminated and explained to others. For example, it may be written up in a paper or thesis, or a conference paper is presented to an audience of conference delegates, or a computer-based product is demonstrated to clients, users or examiners. It is important that the presentation is carried out professionally – otherwise your audience might assume your whole research project was not undertaken in a professional manner."
The presentation of this research will be presented in two deliverable documents, specialization project thesis and master thesis, and a computed-based product\cite{forskerhaandboken}.

\subsection*{References}
	\bibliographystyle{plain}
	\bibliography{sources} 

